\chapter{Fazit}
In diesem Versuch wird ein Helium-Neon-Laser aus einzelnen optischen Komponenten aufgebaut und systematisch auf seine physikalischen Eigenschaften analysiert. Die Wellenlänge der einfallenden Laserstrahlung wird mittels eines Transmissionsgitters bestimmt, während der Polarisationszustand mittels eines Polarisationsfilters untersucht wird. Das Strahlprofil im Resonator wird gemessen und mit den theoretisch erwarteten Modenstrukturen verglichen.

Um Rückreflexionen aus dem Strahlengang zu vermeiden, wird eine optische Diode, bestehend aus einem Polarisator und einer λ/4-Platte, verwendet. Zusätzlich wird der Modenabstand des Lasers sowohl mit einem optischen Spektrumanalysator als auch mit einer Kombination aus einer Photodiode und einem elektrischen Spektrumanalysator gemessen. Diese Messungen werden bei zwei verschiedenen Resonatorlängen durchgeführt, um ihren Einfluss auf das Modenspektrum zu untersuchen.

Ziel des Experiments ist es, ein tieferes Verständnis des Aufbaus und der Funktionsweise des Lasers zu erlangen und die experimentelle Charakterisierung laserphysikalischer Parameter praxisnah zu vermitteln.