\chapter{Fazit}
In diesem Experiment wurde ein Helium-Neon-Laser erfolgreich aus einzelnen optischen Komponenten aufgebaut und dessen Wellenlänge, Polarisation, Strahlprofil und Modenstruktur bei verschiedenen Resonatorlängen untersucht.

Die Wellenlängenbestimmung erfolgte mit einem Lineal in Verbindung mit einem Transmissionsgitter. Die Methode erwies sich als praktikabel. Der experimentell ermittelte Wert betrug \(\lambda = (644,47 \pm 3,32)\)nm, was einer Genauigkeit von etwa 1,8\% entspricht.

Die Bestimmung des Polarisationsgrades mit einem Polarisationsfilter und einer Photodiode ergab einen Wert von $PG=(0{,}98 \pm 0{,}01)$. Dieser hohe Polarisationsgrad bestätigt die Effizienz der im Gerät integrierten Brewster-Fenster. Geringe Abweichungen von der perfekten linearen Polarisation lassen sich durch die spontane Emission von Neonatomen sowie durch unvermeidliche Imperfektionen in den optischen Komponenten erklären.

Bei der Messung des Resonatorstrahlprofils wurden signifikante Abweichungen von den theoretischen Erwartungen beobachtet. Der theoretische Zusammenhang zwischen Strahltaille und Rayleighlänge konnte durch die Messergebnisse bestätigt werden. Das charakteristische Verhalten eines Gaußschen Strahls und die qualitative Ausbreitung des Strahls in Abhängigkeit von der Position relativ zur Strahltaille konnten eindeutig nachgewiesen werden.

Mit dem konfokalen Fabry-Perot-Analysator ließ sich der longitudinale Modenabstand qualitativ korrekt über das Verhältnis $b/a$ bestimmen: Für $l = 68\,\mathrm{cm}$ und $l = 80\,\mathrm{cm}$ stimmten die resultierenden $240 \pm 12\,\mathrm{MHz}$ bzw.\ $180 \pm 12\,\mathrm{MHz}$ innerhalb der Fehlergrenzen mit den theoretischen $220\,\mathrm{MHz}$ bzw.\ $187\,\mathrm{MHz}$ überein und bestätigten das erwartete $1/l$-Gesetz. Dagegen zeigte die 52-cm-Messung einen um $25 \%$ zu großen Wert $360\,\mathrm{MHz}$ statt der erwarteten $288\,\mathrm{MHz}$, was sich auf eine falsche Peakwahl beim Cursor-Setzen oder eine nicht eindeutig definierte Clustergrenze zurückführen lässt. Die optische Methode ist damit zwar anschaulich, bleibt aber anfällig für nichtlineare Piezo-Sweeps und heuristische Cursor-Fehler - ihre Genauigkeit liegt im Bereich einiger $10 \%$.

Die RF-Frequenzmischung erreichte prinzipiell MHz-Auflösung, offenbarte jedoch, wie empfindlich das Verfahren auf perfekte Dual-Mode-Bedingungen reagiert. Bei $l = 70{,}8\,\mathrm{cm}$ stimmen die $m=1$- und $m=2$-Schläge mit $219{,}2\,\mathrm{MHz}$ bzw.\ $220{,}9\,\mathrm{MHz}$ auf unter $4 \%$ mit dem Idealwert überein; ebenso umschließen die beiden 80-cm-Punkte den Sollwert von $187\,\mathrm{MHz}$ innerhalb von $7 \%$. Die Ausreißer bei 52 cm ($303{,}7\,\mathrm{MHz}$) und 68 cm ($135{,}7\,\mathrm{MHz}$ bzw.\ $154{,}9\,\mathrm{MHz}$) wurden erst im Nachhinein als Folge einer dritten, nur 6 dB schwächeren longitudinalen Mode erkannt, die zusätzliche Beat-Linien erzeugt und den gemessenen Schwerpunkt verschiebt. Erst wenn der Analysator unmittelbar vor jeder RF-Aufnahme exakt zwei gleichhohe TEM$_{00}$-Spitzen zeigt und IF-Signale mit FWHM $> 300 kHz$ verworfen werden, rücken alle Datenpunkte in das $\pm 5 \%$-Band um $c/(2l)$.

Die gewichtete Ausgleichsgerade von $\Delta\nu_{\mathrm{exp}}$ gegen $1/(2l)$ liefert $c_{\mathrm{fit}} = (2{,}788 \pm 0{,}007)\times10^{8}\,\mathrm{m/s}$ und damit eine Abweichung von $9.3 \%$ zum CODATA-Wert. Die exorbitante $\chi^{2}/\mathrm{dof}\approx2\times10^{5}$ belegt, dass die kontaminierten Kurz-Kavitäten jeden Gauß-Fehleransatz sprengen; entfernt man diese Punkte oder wiederholt sie unter strikt dualen Bedingungen, konvergiert der Fit auf $c\approx2{,}99\times10^{8}\,\mathrm{m/s}$ und unterschreitet die $1 \%$-Grenze. Für eine weitere Präzisionssteigerung empfiehlt sich (i) GPS-referenzierte Frequenzzählung (< 1 kHz), (ii) Längenerfassung mit Mikrometerschlitten ($\pm 0{,}05 cm$) und (iii) systematische Unterdrückung zusätzlicher Moden durch intracavitäre Etalon oder feinjustierte Spiegelkippung. Damit ist absehbar eine Bestimmung von $c$ auf dem $10^{-4}$-Niveau möglich, womit das Praktikum nicht nur das theoretische $c=2l\Delta\nu_{\mathrm{laser}}$ bestätigt, sondern auch ein metrologisch anspruchsvolles Resultat liefert.
