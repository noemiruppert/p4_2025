\chapter{Fazit}
In diesem Experiment wurde ein Helium-Neon-Laser erfolgreich aus einzelnen optischen Komponenten aufgebaut und dessen Wellenlänge, Polarisation, Strahlprofil und Modenstruktur bei verschiedenen Resonatorlängen untersucht.

Die Wellenlängenbestimmung erfolgte mit einem Lineal in Verbindung mit einem Transmissionsgitter. Die Methode erwies sich als praktikabel. Der experimentell ermittelte Wert betrug \(\lambda = (644,47 \pm 3,32)\)nm, was einer Genauigkeit von etwa 1,8\% entspricht.

Die Bestimmung des Polarisationsgrades mit einem Polarisationsfilter und einer Photodiode ergab einen Wert von $PG=(0{,}98 \pm 0{,}01)$. Dieser hohe Polarisationsgrad bestätigt die Effizienz der im Gerät integrierten Brewster-Fenster. Geringe Abweichungen von der perfekten linearen Polarisation lassen sich durch die spontane Emission von Neonatomen sowie durch unvermeidliche Imperfektionen in den optischen Komponenten erklären.

Bei der Messung des Resonatorstrahlprofils wurden signifikante Abweichungen von den theoretischen Erwartungen beobachtet. Der theoretische Zusammenhang zwischen Strahltaille und Rayleighlänge konnte durch die Messergebnisse bestätigt werden. Das charakteristische Verhalten eines Gaußschen Strahls und die qualitative Ausbreitung des Strahls in Abhängigkeit von der Position relativ zur Strahltaille konnten eindeutig nachgewiesen werden.

!!!!Zur Bestimmung des transversalen Modenabstands wurden zwei Methoden eingesetzt: die Spektralanalyse mit einem optischen Resonator und die Frequenzmischung eines Radiofrequenzgenerators mit einem schnellen Photodiodensignal. Insbesondere die letztgenannte Methode erwies sich als sehr genau, wie der geringe relative Fehler von etwa ??? \% bei der Bestimmung der Modenfrequenzen zeigt. Demgegenüber ergab die Auswertung mittels Spektralanalyse deutlich größere Abweichungen, und die Differenz zur theoretischen Erwartung war etwa von 25\,\%.!!!!!

!!!!!Schließlich wurde durch die genaue Bestimmung des Modenabstands mittels Frequenzmischung ein experimenteller Wert für die Lichtgeschwindigkeit ermittelt. Dieser Wert, $c_{exp}=(\pm) m/s$, lag innerhalb einer relativen Abweichung von weniger als 1 \% des in der Literatur angegebenen Wertes.!!!!!