\chapter{Die Balmer-Serie}
\section{Grundlagen}
\subsection{Bohrsche Atommodel}
Im Bohrschen Atommodell bewegen sich die Elektronen auf bestimmten Kreisbahnen. Diese Kreisbahnen sind ein ganzzahliges Vielfaches der De Broglie Wellenlänge.
Wegen dem Kräftegleichgewichts zwischen Coulombkraft und Zentripetalkraft ergeben sich für die Radien der Kreisbahnen der Elektronen:
\begin{equation}
    r = \frac{n^2 h^2 \epsilon_0}{\pi \mu Z e^2} = \frac{n^2}{Z} a_0
\end{equation}\\
Wobei h das Plancksche Wirkungsquantum, $\mu$ die reduzierte Masse des Elektrons, Z die
Ladungsmenge, e die Elementarladung, n$\in \mathbf{N}$, $a_0$  der Bohrsche Radius ist.
\subsection{Balmer-Serie}
Elektronen können ihre Bahn wechseln, indem sie Energie in Form elektromagnetischer Wellen mit der Frequenz $\nu$ absorbieren oder emittieren, wobei ihre Energie durch $E = h\nu$ bestimmt wird. Die Lichtfrequenz während der Anregung oder Abregung eines Elektrons von einem Energieniveau zu einem Niveau folgt dem Zusammenhang:
\begin{equation}
    \nu = R_{\infty} c Z^2 \left( \frac{1}{n^2} - \frac{1}{m^2} \right)
\end{equation}\\
Dadurch wird die Energiedifferenz zwischen den beiden Zuständen mit Quantenzahlen n und m gegeben. Wobei $R_{\infty}$ der Rydberg-Konstante entspricht.\\
Für ein Wasserstoffatom gilt Z = 1 und mit n = 2 und $m> 2$ und man erhält die Balmerserie an Frequenzen die beobachtet werden und welche im sichtbaren Bereich sind.
\subsection{Quantenmechanische Betrachtung der Balmer-Serie}
\subsection{Isotopieaufspaltung}
Ein Isotop von Wasserstoff is Deuterum: sein Kern besteht aus einem Proton und einem extra Neutron.
Bei der Rydberg- Konstante muss dies berücksichtigt werden. Daraus folgt eine Verschiebung der Energie des emittierten Lichtstrahls. Bei diesen Kernen weißt sich der relative Massenunterschied besonders groß auf, wodurch die Verschiebung besonders deutlich sichtbar ist.
\subsection{Natürliche Linienbreite und Linienverbreiterungen}
\subsection{Reflexionsgitter}
\paragraph{Funktionsweise}
\paragraph{Auflösevermögen}

\section{Aufbau}

\section{Durchführung}

\section{Bestimmung der Gitterkonstanten}

\section{Bestimmung der Balmerlinien}
\subsection{Bestimmung der Isotopieaufspaltung}
\paragraph{Ermittelung des Peakschwerpunktes}
\paragraph{Ermittelung der Halbwertsbreite}
\subsection{Bestimmung der Rydberg-Konstante und des Plankschen Wirkungsquantum}
\paragraph{Rydberg-Konstante}
\paragraph{Planksche Wirkungsquantum}

\section{Weitergehende Überlegungen}
\subsection{Möglicher Ursprung der anderen auftrennen Spektrallinien}
\subsection{Doppler-Verbreitung}
\subsection{Auflösevermögen des Gitters}
