\chapter{Die Balmer-Serie}

Das Bohrsche Atommodel beschreibt ein Atom als einen Kern, mit Elektronen die sich auf bestimmte Kreisbahnen/Energieniveaus um den Kern Bewegen.
Durch hinzufügen von Energie , sowie Photonenabsorbtion oder durch andere äußere Kräfte, können diese Elektronen angeregt werden, welche nun ein höheres Energieniveau haben. 
Um auf ein niedrigeres Energieniveau zurückzukehren muss dieses Elektron Energie in der Form eines Photonen abgeben. 
Diese Energie entspricht der Differenz zwischen dem angeregten Niveau m und Endniveau n, wobei m > n, sehe \cref{fig:Serien der Emissionslinien}. 
Es gibt für jeden Übergang einen bestimmten Namen, zum Beispiel die Lyman-Serie, Balmer-Serie und Paschen-Serie.


%hier auch anders formulieren
Diese Serien sind aber nicht alle sichtbar, die Lyman-Serie strahlt nähmlich im ultravioletten Bereich, und ab der Paschen-Serie sind die Emissionslinien im infrarotem Bereich.
Dazwischen liegt die Balmer-Serien, die ihre Emissionslinien im sichtbaren Bereich hat.
In diesem Versuchsteil sollen die Emissionslinien der Balmer-Serien untersucht werden. 
Hier wird zuerst die Gitterkonstante des benutzten Reflextionsgitter experimentell bestimmt und anschließend sollen die Rydberg-Konstante und das Planksche-Wirkungsquantum anhand von einem Wasserstoffatom bestimmt werden.
Zusätzlich sollen die Emissionslinien von der Deuterium-Lampe untersucht werden und die Genauigkeit der Ergebnisse mit Litteraturwerten verglichen werden.


\section{Aufbau}

Es wurde fogender Versuchsaufbau von \cref{fig:balmeraufbau} verwendet. 

\begin{figure}[htbp]
    \centering
    \includegraphics[width=0.7\linewidth]{figs/aufbau_balmer_serie.png}
    \caption{Versuchsaufbau mit Okular(links) und CCD-Kamera(rechts) \ref{praktikum}}
    \label{fig:balmeraufbau}
\end{figure}
Diese ist wie folgt aufgebaut: 

Es befindet sich eine Deuterium-Lampe (\textbf{a}), dessen Licht durch eine Sammellinese (\textbf{b}) mit Brennweite $f = 50mm$ auf ein verstellbareren Spalt (\textbf{c}) abgebildet wird, damit der einfallende Lichtstrahl begrenzt wird. 
Hinter dem Spalt befindet sich ein Projektionsobjektiv (\textbf{d}), mit Brennweite $f = 150mm$. 
Dieses soll genau im Abstand seiner Brennweite zum Spalt stehen, damit der Lichtstrahl parallel zum holographischen Gitter (\textbf{e}) einfällt. 
Dieses holographische Gitter ist ein Reflextrionsgitter welches sich auf der drehbaren Säule des Drehgelenk befindet und genutzt wird, um die Spektrallinien der Lampe aufzuspallten. 
Das reflektierte Licht wird anschliesend mit einer Sammellinse (\textbf{f}) der Brennweite $f=300mm$ auf einem Okular (\textbf{g}) abgebildet.
Das Okular kann alternativ mit einer CCD-Kamera (\textbf{h}) für genauere Messungen ersetzt werden.

\section{Durchfühung}

\paragraph{Justierung}

Um die Gitterkonstante zu bestimmen, muss von einem bekannten Element die Spektrallinien untersucht werden. 
Dazu wird die Deutrium lampe (Balmer-Lampe) mit einer Quecksilder Lampe (Hg-Lampe) ersätzt. 
Hierzu muss darauf geachtet werden das alle Bauteile des Aufbaus auf der gleichen hohe bleiben, damit es keine veränderungen der optischen achse mit der Balmer-Lampe geben würde.
Es wird nun die Linse \textbf{b} so justiert, das es einen scharfen Lichtfleck von der Lampe auf der Platte abgebildet wird.
Das Projektionsobjektive \textbf{d} wird auf ungefähre Brennweite hinter den Spalt positioniert. 
Es wird nun das Drehgelenk des Gitters (\textbf{e}) auf die 0$^\circ$ position gebracht und das Projektionobjektive so verschoben, dass ein schafes Bild des Spaltes auf dem spalt erkännbar ist, so wird der SPalt im unendlichen abgebildet.
Zuletzt wird die Linse \textbf{f} so justiert, dass im Okular ein scharfes Bild, im Spektrum, zu erkennen ist. Dieses Bild soll eine beliebige Spektrallienie der ersten Ordnung sein.
Nun soll, für den folgenden Versuchsteil, die Winkel des optischen Bank ($\omega_B$) und das Winkel des Gitters ($\omega_G$) abgelesen werden.  
Damit diese werte benutzt werden können, müssen diese in die Relewanten winkel für das Gitter umgerechnet werden, sehe \cref{Gitter Balmer}. 
Mit Hilfe von \cref{Gitter Balmer} können 
\begin{align}
    \alpha &= \omega_G \\  \beta &= \omega_B + \omega_G - 180^\circ 
\end{align}
Dieses Soll nach dem Zurücktausch der Hg-Lampe und der Balmer-Lampe Widerholt werde. 


\paragraph{Bestimmung der Gitterkonstante}
<<<<<<< HEAD
=======
<<<<<<< HEAD
=======
>>>>>>> 068157510707739f27696c762eece109a9f399eb

>>>>>>> abt
Die Gitterkonstent wird mithilfe der Hg-Lampe bestimmt.
Es wird nach dem ersten Spektral linie gesucht, bis dies gefunden ist. 
Hier zu wird die hälichkeit der Spektrallinie über die Spalt aufgedreht, wenn diese nicht sichtbar sind und dann auf etwa 1 Skalen Teil (0,1mm) eingestellt, aber dass die nicht verschindet.
Um zu vergleichen welche Wellenlänge gesehen werden konnte für die auswertung wurde die Hg-Linien von dem Anhang \cref{Hg-Linien} zu nutzen.
Es werden nun $\omega_B$ und $\omega_G$ abgelesen und mit \cref{Hg-Linien} zugeordnet.

\paragraph{Untersuchung der Balmer-Linien}
Nach dem Tauschen der Lampen wird erstmal die Justierung Widerholt. 
Nach der Justierung werden für jede Spektrallinie Widerrum die Winkel $\omega_B$ und $\omega_G$ gemessen und der Abstand der Aufspaltung $d$, der SPektrallinien, abgeschätzt.

\paragraph{Ersätzen Okular mit CCD-Kamera}
Es wird nun das Okular mit einer CCD-Kamera ersätzt, damit eine genauere bestimmung der Spektrallinien stadt finden kann. 
Es wird ein Programm genutzt, welches die Intensität und Pixelkoordinate (Position der Intensität) aufnimmt und gegen einander aufträgt. 
Falls die Intensität zu klein ist, kann im programm die Schaltfläche vergrößert werden.
Das Programm gibt aber einen Winkel aus , den Ausfalls winkel, welches es aus den Pixelkoordinaten entnimmt, mit 
\begin{equation}
    \beta = \arctan(\frac{(1024-p)\cdot0,014mm}{f})
\end{equation}
wobei, $p$ die Pixelkoordinate und $f$ die Brennweite der abbildenden Sammellinse sind.
Diese Linse wird noch verschoben, bis die Darstellung des Programmes Scharf dagestellt werden kann (die Peaks sollen so dünn wie möglich sein).
Da die Intensität sehr sensitive ist, wird mit Hilfe des Programms einen Mittelwerts Bildung der Intensität gemacht. 
Diese Werte werden gespeichert und die Winkel des Gitter aufgenommen. 
Es wird der gleich Vorgang für die weitere Balmer-Linien Verhandet.

\section{Bestimmung der Gitterkonstanten}
Um die Gitterkonstante zu berechnen wird die Gitter gleichung für ein Reflexionsgitter 
\begin{equation}
  g\bigl(\sin\alpha + \sin\beta\bigr) = m\,\lambda
  \quad\Longrightarrow\quad
  g = \frac{m\,\lambda}{\sin\alpha + \sin\beta}
  \label{Gittergleichung}
\end{equation}
genutzt, mit m die Ordnung, $\lambda$ die Wellenlänge, $\alpha$ der Einfall Winkel und $\beta$ der Ausfalls Winkel, mit fehler
\begin{equation}
  \Delta g
  = \sqrt{
    \Bigl(\tfrac{\partial g}{\partial\alpha}\,\Delta\alpha\Bigr)^{2}
   +\Bigl(\tfrac{\partial g}{\partial\beta}\,\Delta\beta\Bigr)^{2}
  }.
  \label{Gitterfgleichung fehler}
\end{equation}
\begin{equation}
  \frac{\partial g}{\partial\theta_m}
  = \frac{m\,\lambda\,\cos\theta_m}{(\sin\theta_m + \sin\beta)^{2}},
  \quad
  \frac{\partial g}{\partial\beta}
  = \frac{m\,\lambda\,\cos\beta}{(\sin\theta_m + \sin\beta)^{2}}.
\end{equation}
\begin{equation}
    \Longrightarrow \Delta g = \sqrt{\Bigl(\tfrac{m\,\lambda\,\cos{\alpha}}{(\sin{\alpha}+\sin{\beta})^2}\,\Delta\alpha\Bigr)^2 + \Bigl(\frac{m\,\lambda\,\cos{\beta}}{(\sin{\alpha} + \sin{\beta})^2}\,\Delta\beta\Bigr)^2}
\end{equation}

Es wird m = 1 gesetzt, da dies die ordnung ist, die untersucht wird.


Dies ausgerechneten Werte befinden sich in \cref{tab:gitterkonstante} mit den Entsprächenden abhängigen werte und dessen fehler.
\begin{table}[htbp]
    \centering
    \begin{tabular}{|c|c|c|c|c|}
        
         &  \\
         & 
    \end{tabular}
    \caption{Caption}
    \label{tab:my_label}
\end{table}
Es ist zu bemerken, dass für die roten Spektrallinien für $\omega_B = 145^\circ$ nicht sichtbar waren wurde diese geändert und zur Überprüfung, schon gemessene Spektrallinien nochmals ausfgenommen. 
Zu beachten, ist dass diese Werte nicht genau übereinstimmen, was mit schlechtem abschätzen zu tun haben könnte, da zum Beispiel $61,2^\circ$ und $61,0^\circ$ kaum zu unterscheiden waren.
Mit der Annahme dieses Fehlers sind die Werte angemessen.
Zusätzlich waren manche Linien so blass, das diese kaum erkannt wurden und mehr Linien gesehen wurden. 
Diese wurden aber nicht genommen, da diese sehr schlecht zu sehen waren. 
Um einen festen Wert zu haben um für die Balmer-Linien zu berechnen, wurde der mittelwert von den ausgerechneten Gitterkonstanten genommen mit 

\begin{equation}
  \overline{g}
  = \frac{\sum_{i=1}^{N} \bigl(g_i/\Delta g_i\bigr)}
         {\sum_{i=1}^{N} \bigl(1/\Delta g_i\bigr)},
  \quad
  \Delta\overline{g}
  = \sqrt{\frac{N}{\sum_{i=1}^{N} 1/(\Delta g_i)^{2}}}\,.
\end{equation}

Es ergibt sich nun die Gitterkonstante mit:
\begin{equation}
    \overline{g} = (420.76 \pm 1.51) nm.
    \label{gitterkon}
\end{equation}
Dieser Wert passt nicht zu allen g-Werte, aber mit mehr als $2/3$ und ist somit ein sinvoller Wert. 

\section{Bestimmung der Balmerlinien}

Mit der Gitterkonstante kann nun die Wellenlängen der Balmer-Lampe berechnet werden. 
Dies kann durch die Gittergleichung \ref{Gittergleichung} mit der Ersten Ordnung berechnet werden.
Die dazu gehörige Wellenlängen ist zwischen 388nm und 656nm sichtbar ([Uni Ulm]) und mit den Messung zuzuordnen.
Die Emissionslinien sind dabei die Übergänge von Energieniveau $n > 2 \xrightarrow{} n = 2$ 
Die Photonen die den Übergang beschreiben, kann durch die Rydberg-Formel (\cref{[Demtröder_Ex3]}, S.100) 
\begin{equation}
  \frac{1}{\lambda}
  = Ry\Bigl(\tfrac{1}{2^{2}} - \tfrac{1}{n^{2}}\Bigr),
  \quad n=3,4,5,\dots
  \label{Rydberg-Formel}
\end{equation}
gezeigt werden.
Diese wird noch im \cref{Rydberg-konst} bestimmt. 
Es wird nochmals die Winkel für die bestimmten Emissionslinien aufgenommen werden und die Ausgerechneten Werte in \cref{tab: gesehenes deut}, so wie deren Literautrwert aufgelistet. 

Wärend des versuches, wurden nur 3 Emissionslinien gesichtet, dies könnte an dem Fehlenden Abschirmung der Lampe liegen könnte welches durch Reflextion an der linse vor dem Okular, die schwer zu sehenden Emissionslinien, Überleuchtet hat. %Versuche zu sagen, dass wegen das licht der lampe die linie nicht zu sehen ist.
Dieses könnte zu H$_\alpha$, H$_\beta$ und H$_\gamma$ zugeordnet werden.

\begin{table}[htbp]
    \centering
    \begin{tabular}{c|c}
         &  \\
         & 
    \end{tabular}
    \caption{Deuterium}
    \label{tab: gesehenes deut}
\end{table}

Es ist zu sehen, dass die berechneten Wellenlängen nicht mit den Literaturwert übereinstimmt. 
Dies könnte an der näherung der Winkel liegen, da diese zum Beispiel als $55,3^\circ \approx 55,5^\circ$.
Zusätzlich hätten die Fehler auch zu klein Geschetzt werden können.
Obwohl die Werte nicht mit den Fehler mit den Literaturwerte übereinstimmen, sind die Werte genau genug, um die Werte zuzuordnen. 


\subsection{Bestimmung der Isotopieaufspaltung}

Bei der Unteruchung der Emissionslinien der Balmer-linien, wurde gesehen, dass die Emissionslinien eine zweite Emissionslinie existiert.
Der Grund hierfür ist an der Balmer-Lampe. 
<<<<<<< HEAD
Da diese nicht rein aus Deuterium, sonder auch Wasserstoff besteht, im Verhältnis von $\approx 1 : 2$ ([Praktikum]). 
=======
<<<<<<< HEAD
Da diese nicht rein aus Deuterium, sonder auch Wasserstoff besteht, im Verhältnis von $\approx 1 : 2$ ([Praktikum]). 
=======
Da diese nicht rein aus Deuterium, sonder auch Wasserstoff besteht, im Verhältnis von $\approx 1 : 2$ (praktikum). 
>>>>>>> abt
>>>>>>> 068157510707739f27696c762eece109a9f399eb
Dies Weist darauf hin, das die Kern Masse einen Einfluss auf die Energieniveaus hat. 
Aus der Quantenmechanik kann die Rydberg-konsante zusätzlich mit 
\begin{equation}
  R_{\mathrm{y}} = \frac{\mu\,e^4}{8\,c\,\epsilon_0^2\,h^3}
\end{equation}
beschrieben werden(\cref{[Demtröder_Ex3]}, S.101). Dabei ist zu beachten, dass dieser wert von der Reduzierten Masse abhängt.
Durch 
\begin{equation}
    \mu = \frac{m_e \cdot m_K}{m_e + m_K} = \frac{m_e}{1+\frac{m_e}{m_K}}
\end{equation}
mit $m_e$ die Elektronen Masse und die Kernmasse $m_K$.
Somit kann ein fester Rydbergkonstante ($Ry_\infty$) bestimmt werden: 
\begin{equation}
    Ry = \frac{1}{1+\frac{m_e}{m_K}}\cdot \frac{\mu m_e e^4}{8c \epsilon_0^2h^3} = \frac{1}{1 + \frac{m_e}{m_K}}\cdot Ry_\infty
\end{equation}
Da das Deuterium einen extra Neutron hat ist dieses Schwärer, somit ist die Ry kleiner und so auch proportional die Wellenlänge. 
Dieses wurde auch für größere Wellenlänge deutlicher sichtbar.
Diese Aufspeltung wird als Isotopieaufspaltung bezeichnet, wobei es sich in diesem fall über ein Masseneffekt der Isotopiaufspaltung handelt.

Mit der Skala in dem Okular kann die Größe d der Isotopieaufspaltung für die Emissionslinien geschätzt werden. 
Diese befinden sich in \cref{tab:Isotopie}.

Dies kann durch die \cref{Gittergleichung}
\begin{equation}
  \lambda = g\,(\sin\alpha + \sin\beta),
  \quad
\frac{\Delta\lambda}{\Delta\beta} \approx 
  \frac{\partial\lambda}{\partial\beta} = g\,\cos\beta,
  \quad
  \Delta\beta \approx \frac{d}{f},
\end{equation}
und mit der Brennweite der Abbildungslinse kann sich der Winkel $\Delta\beta$ durch 
\begin{equation}
    \Delta\beta = \arctan\Bigl(\frac{d}{f}\Bigr) \approx \frac{d}{f} \quad \text{für } d \ll f
\end{equation}
berechnen lassen.

Diese Werte sind aber nicht genau, da diese aufspaltung sehr schwer zu sehen war und nur mit mühe versucht abzuschätzen.

Die CCD-Kamera hat dieses Problem aber nicht und kann genauer die Isotopieaufspaltung Messen.

Die Gemessenen Intensitäten Bilden Peaks die in \cref{fig:H_a}, \cref{fig:H_b} und \cref{fig:H_g} dargestellt sind. 
\begin{figure}
    \centering
    \includegraphics[width=0.5\linewidth]{figs/dt_lila_145_51_5.png}
    \caption{Isotopieaufspaltung von der $H_{\gamma}$}
    \label{fig:H_g}
\end{figure}

Hierbei sind mehrere Peaks zu erkennen und können durch folgende Gauß-Peak Funktion 
\begin{equation}
    I(\beta) = \sum_i^2 A_i \cdot \exp{\Bigl(-\frac{\beta-\mu_i}{2\sigma_i}}\Bigr) +b
\end{equation}
berechent werden, mit b, dem Offset und dem Winkel $\beta$. 

Die Berechneten werte sind in \cref{tab:Peaks} zu sehen.
Die Isotopieaufspaltung dessen, kann durch 
$\Delta\beta = |\mu_2 - \mu_1|$
beschrieben werden. 
diese sind:
\begin{table}[htbp]
    \centering
    \begin{tabular}{c|c}
         &  \\
         & 
    \end{tabular}
    \caption{Caption}
    \label{tab:my_label}
\end{table}
Obwohl nur 3 Emissionslinien gesehen wurden, hat die CCD- Kamera noch eine zusätzlich aufgenommen, welches als $H_\delta$ vermutet wird.
Die Berechneten Werte weichen aber Significant ab.
Dies lag vermutlich daran, dass die Balmer-Lampe nicht abgeschirmt war und somit die Auswertung beeinträchtigt. 
Es kann aber trotzdem gesehen werden, was es eine Aufspaltung gibt und somit das Ziel der Untersuchung erreicht.

\subsection{Bestimmung der Rydberg-Konstante und des Plankschen Wirkungsquantum}
\paragraph{Rydberg-Konstante} \label{Rydberg-konst}
Wie schon in abschnitt \ref{Bestimmung der Balmer} erwähnt kann die Rydberg-konstante über die wellen länge bestimmt werden.
Dafür ist eine folgende umrechnung nötig:
\begin{equation}
  Ry
  = \frac{1/\lambda}{\bigl(\tfrac{1}{4} - \tfrac{1}{n^{2}}\bigr)},
  \quad
  \Delta Ry
  = \frac{\Delta\lambda}{\lambda^{2}\,\bigl(\tfrac{1}{4} - \tfrac{1}{n^{2}}\bigr)}.
\end{equation}
Mit den Berechneten Werten in \cref{tab:Rydberg} kann gesehen werden, das die Werte miteinander Übereinstimmen, haben aber nicht den gleichen wert. 

\begin{table}[htbp]
    \centering
    \begin{tabular}{|c|c|c|c|}
        Linie & $\lambda$ / nm & n & Rydberg-Konstante /$10^7$ m \\
        \hline
        $H_\alpha$ & 620,049 $\pm$ 4,091 & 5 & 1,161 $\pm$ 0,008 \\
        $H_\beta$ & 494,113 $\pm$ 4,393 & 4 & 1,079 $\pm$ 0,010  \\
        $H_\gamma$ & 442,969 $\pm$ 4,508 & 3 & 1,075 $\pm$ 0,011 \\
    \end{tabular}
    \caption{Berechneten Rydbergkonstanten für die berechentn Wellenlängen}
    \label{tab:Rydberg}
\end{table}

Um einen Wert zu haben wird das gleiche verfehren benutzt wie bei der Gitterkonstante:
\begin{equation}
  \overline{g}
  = \frac{\sum_{i=1}^{N} \bigl(g_i/\Delta g_i\bigr)}
         {\sum_{i=1}^{N} \bigl(1/\Delta g_i\bigr)},
  \quad
  \Delta\overline{g}
  = \sqrt{\frac{N}{\sum_{i=1}^{N} 1/(\Delta g_i)^{2}}}\,.
\end{equation}
Dies liefert einen Wert von 
\begin{equation}
    Ry = (1,105 \pm 0,006)\cdot 10^7 \frac{1}{m}.
\end{equation}
Dieser Wert stimmt sehr gut mit dem Literatur Wert (\cref{[Demtröder_Ex3]}, S.101) von der Rydbergkonstante welches 
\begin{align}
    Ry_{lit} = 1,097 \frac{1}{m},
\end{align}
obwohl es nicht in dem Fehler liegt, da dieser sehr klein ist, liegt der Literaturwert inerhalb von dem doppelten fehler, was für einen kleinen Fehler, von >1$\%$, sehr gut ist.
Mir könnte auch einen Felher in der Fehlerrechnung vorgekommen sein.

\paragraph{Planksche Wirkungsquantum}

Da die Rydberg-Konstante ausgerechet wurde, kann hierraus das Planksche wirkungsquantum bestimmt werden:

\begin{equation}
    Ry = \frac{\mu e^4}{8 c \epsilon_0^2h^3}
\end{equation}
\begin{equation}
  \Leftrightarrow h
  = \Bigl(\tfrac{m_e\,e^{4}}{8\,\varepsilon_{0}^{2}\,c\,Ry}\Bigr)^{1/3},
  \quad
  \Delta h
  = \frac{1}{3}
    \Bigl(\tfrac{m_e\,e^{4}}{8\,\varepsilon_{0}^{2}\,c}\Bigr)^{1/3}
    Ry^{-4/3}\,\Delta Ry.
\end{equation}

Somit Wurde das Planksche Wirkungsquantum als
\begin{equation}
    h = (6.6104 \pm 0.109)\cdot 10^{-34} J\cdots
\end{equation}
berechent was mit dem Literaturwert (\cref{[Demtröder_Ex3]}, S.75) von
\begin{equation}
    h = 6,626 \cdot10^{-34} J\cdot s
\end{equation}
sehr gut über ein stimmt.
Dies liegt auch innerhalb des doppelten fehler und und stimmt auch mit der Messung der Plankschen Wirkungsquantum Photoeffekt überein.


\section{Weitergehende Überlegungen}
\subsection{Möglicher Ursprung der anderen auftrennen Spektrallinien}
In dem Versuchsteil gibt es aber nicht nur die Balmer-Linien die beobachtet werden kann. 
Durch einfallendes Licht von anderen Quellen, vor allem Elektronische Geräte die Benutzt werden um die Messung aufzunehmen, gibt es die möglichkeit das diese Aufgenommen werden könnte.
Sehr wichtig ist, dass der EffeKt des Streulichtes berücksichtigt wird, was die Bamler-lampe abgibt, da dieses anders Reflektiert wwerden kann, als wenn es mittig auf dem Gitter trifft.
Wenn die Linsen nicht in Ordnung gehalten worden wären, aber auch durch die Zeit, würden zusätzlich Linsenfehler oder Aberrationen auftreten und muss somit auch berücksichtigt werden. 
Zusätzlich muss beachtet werden das es mehr als eine Ordnung gibt und somit die Letzten beobatbaren Linien der Ordnung mit Linien einer Höheren Ordnung Überschnieden kann.

\subsection{Doppler-Verbreitung}

Wegen der Termischen Bewegung der Atome relative zu einem ruhenden Betrachter, entsteht eine Doppler-Verschiebung des emitierten Photonen. 
Dies führ zu einer verbreiterung der Spektrallinien über ihre natürliche Breite hinaus, welche mit der Formel (\cref{[Demtröder_Ex3]}, S.230)
\begin{equation}
    \delta\lambda = \frac{\lambda_0}{c}\cdot \sqrt{\frac{8k_BT\cdot ln2}{m}}
\end{equation}
beschrieben werden kann.
Dabei ist $\delta\lambda$ die Halbwertsbreite, die Temperatur $T \approx 1000K$ ([praktikum]) und m die masse des Atomes (2$m_H \approx m_D$)
Diese Werte werden in \cref{tab:dopplerTemp}
\begin{table}[htbp]
    \centering
    \begin{tabular}{|c|c|c|c|}
    \hline
    Linie & Literaturwert $\lambda$ in nm & $\delta\lambda$ für $^1H$ in nm & $\delta\lambda$ für $^2H$ in nm \\
    \midrule
    H$_\alpha$ & 656,28 & 0,015 & 0,011 \\
    H$_\beta$ & 486,13 & 0,011 & 0,008 \\
    H$_\gamma$  & 434,05 & 0,010 & 0,007 \\
    \hline
    \end{tabular}
    \caption{(Theoretische) Doppler-Verbreitung von Wasserstoff und Deuterium}
    \label{tab:dopplerTemp}
\end{table}

Um die natürliche Linienbreite $\delta\nu$ zu berechnen, wird die Formel
\begin{equation}
    \delta \nu = \frac{1}{2\pi \cdot \tau}
\end{equation}
genutzt (\cref{[Demtröder_Ex3]}, S.228f), mit $\tau$ die Lebesdauer. 
Diese Formel nicht von fremden Einflüsse abhängt und kann auch von der Heisenberg'schen Unbestimmtheitsrelation herleiten.  
Da die Lebesdauer von Atomen liegt in der Größenordnung von $10^{-8}s$. 
Dies führt dazu, dass $\delta\nu$ in der größenordnung von $16MHz$  ist und durch $\frac{\Delta\nu}{\nu} = \frac{\Delta\lambda}{\lambda}$ mit $\nu = \frac{c}{\lambda}$ führt dazu, dass die Linienbreit in der Ordnung von $10^{-14}nm$ ist und somit vernachlässigbar groß gegenüber der Dopplerverbreitung ist.
So ist es sinvoller die Doppler-Verbreitung mit den Hablwertsbreiten der berechneten Gaußkurven zu vergleichen.
Dieser kann durch 
\begin{equation}
    \delta\beta = 2\sqrt{\ln{2}}\cdot \sigma
\end{equation}
wobei $\sigma$ die Halbwertsbrete ist.
Hierdurch kann mit \cref{deltalambda} die Wellenlängendifferenz berechnen.
Die daraus bekommenen Werte sind in \cref{tab:Halbwertsb}.

\begin{table}[htbp]
    \centering
    \begin{tabular}{|c|c|c|}
    Linie & $\delta\lambda$ für $^1$H in nm & $\delta\lambda$ für $^2$H in nm \\
    \hline
        H$_\alpha$ & 0.848 ± 0.006 & 0.586 ± 0.049 nm \\
        H$_\beta$ & 2.258 ± 0.051 nm & 0.609 ± 0.003 \\
        H$_\gamma$ & 0.831 ± 0.021 nm & 0.758 ± 0.004 nm
    \end{tabular}
    \caption{Gemessene Halbwertsbreite für bekannte Spektrallinien}
    \label{tab:gemhalb}
\end{table}
Es ist deutlich zu sehen, dass die Halbwertsbreiten sehr groß sind. 
Die fehler sind auch sehr klein, dies kann an einer falschen abschätzung einiger Gaußplots, zum Beispiel bei H$_\beta$ konnte die Gauß Kurve nicht gut zu dem zeiten Peak angepasst werde.
Zusätzlich könnte wolmöglich auch die Linienbreite besser eingeengt werden, da der Aufbau möglicher weise nicht richtig justiert wurde, sodass die Spalt breite zu groß gewählt wurde und nicht vollständig ausgeleuchtet war oder das Streulicht hat die Basis Intensität so sehr angehoben, durch fehlender abschirmung, das die Peaks einbischen ertränkt werden.


\subsection{Auflösevermögen des Gitters}

Anschliessend Wird das Auflösungsvermögen des Gitters abgeschätzt. 
Dies wird mit der Formel 
\begin{equation}
    A = \frac{\lambda}{\Delta\lambda} = m \cdot N
\end{equation}
gemacht, wobei m die Ordnungs Zahl ist und N die Anzahl der Beleichteten Spalten. 
Das geammst Gitter Beträgt eine 25 mm x 25 mm fläche:
\begin{equation}
    \Rightarrow N = \frac{d}{g}, d = 25mm
\end{equation}
\begin{equation}
    \Rightarrow A = \frac{25mm}{420,76nm} = (5,94 \pm 0,02) \cdot 10^4
\end{equation}
So wäre es möglich alle Aufspaltungen zu Messen, da das Auflödungsvermögen von dem Gitter so groß ist. 
Dies ist würde aber bedeuten, dass das ganze Gitter beleuchtet werden müsste, welches aber mit der untersuchung der Isotopieaufspaltung stören würde, da dies nicht mehr sichtbar wäre.
Um dieses Wiederum zu messen zu können müsste der Spalt verringert werden und so mit das licht was auf dem Gitter fallen sollte.
