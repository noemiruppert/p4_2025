\chapter{Einleitung}
%beide Versuchsteile etwas besser erläutern, genauer erklären, wieso wir diesen Versuch überhaupt durchführen
Ein zentraler Versuch zur Bestätigung des Zusammenhangs zwischen der Quantelung von Energien und Emissions -und Absorptionslinien ist die Untersuchung des Photoeffekts. Die Spektroskopie ermöglicht die Untersuchung des Atomaufbaus, insbesondere durch die Analyse von Spektrallinien, welche einen Ausdruck der Quantelung von Energie sind und in direktem Zusammenhang mit Lichtfrequenzen stehen. \\
Im ersten Versuchsteil beobachtet man die Energieabhängigkeit des Photoeffekts und es werden das Planksche Wirkungsquantum, sowie die Austrittsarbeit abgeschätzt. 
\\
Im zweiten Versuchsteil wird durch Ausmeßung der Balmer-Linien das Planksche Wirkungsquantum erneut bestimmt und mit dem Ergebnis aus dem ersten Versuchsteil verglichen.
