\chapter{Formeln: To be deleted at the end}

\section*{Spannungsteiler}

\begin{equation}
  U = \frac{R_2}{R_1 + R_2}\,U_{\mathrm{ges}}
\end{equation}

mit $U$ als Spannung am Widerstand $R_2$, $R_1$ und $R_2$ als Widerstände und $U_{\mathrm{ges}}$ als Gesamtspannung.

\section*{Energieerhaltung}

\begin{equation}
  h\,f = E_{\mathrm{kin}} + W_A,
  \quad
  E_{\mathrm{kin}} = e\,U_G
\end{equation}

mit $h$ dem Planckschen Wirkungsquantum, $f$ der Photonfrequenz, $e$ der Elementarladung, $U_G$ der Gegenspannung und $W_A$ der Austrittsarbeit.

\section*{Fehlerfortpflanzung I}

\begin{equation}
  \Delta\bigl(\sqrt{I - I_0}\bigr)
  = \sqrt{
    \Bigl(\tfrac{\Delta I}{2\,\sqrt{I - I_0}}\Bigr)^{2}
   +\Bigl(\tfrac{\Delta I_0}{2\,\sqrt{I - I_0}}\Bigr)^{2}
  }.
\end{equation}

\section*{Beugungsgitter}

\begin{equation}
  g\bigl(\sin\theta_m + \sin\beta\bigr) = m\,\lambda
  \quad\Longrightarrow\quad
  g = \frac{m\,\lambda}{\sin\theta_m + \sin\beta}
\end{equation}

\begin{equation}
  \Delta g
  = \sqrt{
    \Bigl(\tfrac{\partial g}{\partial\theta_m}\,\Delta\theta_m\Bigr)^{2}
   +\Bigl(\tfrac{\partial g}{\partial\beta}\,\Delta\beta\Bigr)^{2}
  }.
\end{equation}

\begin{equation}
  \frac{\partial g}{\partial\theta_m}
  = \frac{m\,\lambda\,\cos\theta_m}{(\sin\theta_m + \sin\beta)^{2}},
  \quad
  \frac{\partial g}{\partial\beta}
  = \frac{m\,\lambda\,\cos\beta}{(\sin\theta_m + \sin\beta)^{2}}.
\end{equation}

\section*{Mittelwert der Gitterkonstante}

\begin{equation}
  \overline{g}
  = \frac{\sum_{i=1}^{N} \bigl(g_i/\Delta g_i\bigr)}
         {\sum_{i=1}^{N} \bigl(1/\Delta g_i\bigr)},
  \quad
  \Delta\overline{g}
  = \sqrt{\frac{N}{\sum_{i=1}^{N} 1/(\Delta g_i)^{2}}}\,.
\end{equation}

\section*{Isotopenverhältnis}

\begin{equation}
  \lambda = g\,(\sin\theta_m + \sin\beta),
  \quad
  \frac{\partial\lambda}{\partial\beta} = g\,\cos\beta,
  \quad
  \Delta\beta \approx \frac{d}{f}.
\end{equation}

\section*{Fehlerfortpflanzung II}

\begin{equation}
  \Delta\lambda
  = \sqrt{
      \Bigl(\tfrac{\lambda}{g}\,\Delta g\Bigr)^{2}
    + \Bigl(g\,\cos\alpha\,\Delta\alpha\Bigr)^{2}
    + \Bigl(g\,\cos\beta\,\Delta\beta\Bigr)^{2}
  }.
\end{equation}

\begin{equation}
  \Delta(\Delta\lambda)
  = \sqrt{
      \Bigl(\tfrac{d\,\cos\beta}{f}\,\Delta g\Bigr)^{2}
    + \Bigl(\tfrac{-d\,\sin\beta}{f\,g}\,\Delta\beta\Bigr)^{2}
    + \Bigl(\tfrac{g\,\cos\beta}{f}\,\Delta d\Bigr)^{2}
  }.
\end{equation}

\section*{Balmer‐Formel}

\begin{equation}
  \frac{1}{\lambda}
  = R_H\Bigl(\tfrac{1}{2^{2}} - \tfrac{1}{n^{2}}\Bigr),
  \quad n=3,4,5,\dots
\end{equation}

\begin{equation}
  R_H
  = \frac{1/\lambda}{\bigl(\tfrac{1}{4} - \tfrac{1}{n^{2}}\bigr)},
  \quad
  \Delta R_H
  = \frac{\Delta\lambda}{\lambda^{2}\,\bigl(\tfrac{1}{4} - \tfrac{1}{n^{2}}\bigr)}.
\end{equation}

\section*{Plancksches Wirkungsquantum}

\begin{equation}
  h
  = \Bigl(\tfrac{m_e\,e^{4}}{8\,\varepsilon_{0}^{2}\,c\,R_H}\Bigr)^{1/3},
  \quad
  \Delta h
  = \frac{1}{3}
    \Bigl(\tfrac{m_e\,e^{4}}{8\,\varepsilon_{0}^{2}\,c}\Bigr)^{1/3}
    R_H^{-4/3}\,\Delta R_H.
\end{equation}
