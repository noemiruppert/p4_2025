\chapter{Anhang}
\section{Abbildungen}

\section{Tabellen}
\subsection*{Photoeffekt}
%=======================================================================================================================================================================================================
\begin{table}[H]
\centering
\resizebox{\columnwidth}{!}{
\begin{tabular}{|c|c|c|c|c|c|c|c|}
\hline
\multicolumn{4}{|c|}{$\lambda = \SI{365}{\nm}$}
  & \multicolumn{4}{c|}{$\lambda = \SI{405}{\nm}$} \\ \hline
\multicolumn{2}{|c|}{Messung 1} & \multicolumn{2}{c|}{Messung 2}
  & \multicolumn{2}{c|}{Messung 1} & \multicolumn{2}{c|}{Messung 2} \\ \hline
$U_G {[\si{mV}]}$   & $U_{ph} {[\si{mV}]}$  & $U_G {[\si{mV}]}$  & $U_{ph} {[\si{mV}]}$ & $U_G {[\si{mV}]}$    & $U_{ph} {[\si{mV}]}$ & $U_G {[\si{mV}]}$  & $U_{ph} {[\si{mV}]}$ \\ \hline
0,5             & 2380,0            & 0,5           & 2382,0           & 0,5             & 900,0            & 0,5           & 914,0            \\ \hline
30,5            & 2070,0            & 35,6          & 2065,0           & 41,5            & 722,0            & 36,0            & 740,0            \\ \hline
84,2            & 1630,0            & 84,6          & 1624,0           & 49,4            & 680,0            & 53,9          & 665,0            \\ \hline
121,0             & 1325,0            & 120,5         & 1334,0           & 105,4           & 450,0            & 102,0           & 464,0            \\ \hline
152,9           & 1071,0            & 156,3         & 1061,0           & 149,6           & 293,8          & 152,9         & 286,9,0          \\ \hline
160,7           & 1025,0            & 170,3         & 951,0            & 167,7           & 245,0            & 165,6         & 250,5          \\ \hline
182,4           & 883,0             & 180,8         & 892,0            & 174,5           & 226,9          & 178,2         & 215,6          \\ \hline
190,7           & 823,0             & 194,8         & 795,0            & 201,7           & 161,2          & 200,3         & 164,8          \\ \hline
217,9           & 655,0             & 219,4         & 653,0            & 213,7           & 139,5          & 217,5         & 230,0            \\ \hline
270,0             & 396,0             & 271,1         & 392,0            & 247,4           & 86,1           & 254,1         & 78,8           \\ \hline
287,0             & 333,3           & 287,4         & 325,0            & 267,7           & 63,5           & 273,2         & 58,2           \\ \hline
355,7           & 152,7           & 359,2         & 148,8          & 289,6           & 45,8           & 293,0           & 24,8           \\ \hline
396,3           & 79,9            & 397,2         & 95,6           & 301,9           & 38,8           & 304,7         & 36,5           \\ \hline
452,0             & 32,9            & 455,0           & 29,2           & 352,9           & 14,8           & 345,3         & 17,6           \\ \hline
472,0             & 14,1            & 477,0          & 11,7           & 378,1           & 5,1            & 384,8         & 3,2            \\ \hline
517,0             & 1,1             & 516,0           & 1,4            & 410,0             & 1,1            & 405,0           & 1,1            \\ \hline
\end{tabular}%
}
\caption{Messwerte der Photospannung $U_{ph}$ bei Gegenspannung $U_G$ für $\lambda$ = $\SI{365}{\nm}$ und $\lambda$ = $\SI{405}{\nm}$, wobei $\Delta U_{ph} = 0.1 \cdot U_{ph} + 10\,\si{\milli\volt}, \quad
 \Delta U_{G} = 10 \si{\milli\volt}$}
\label{tab:365and405}
\end{table}
%=======================================================================================================================================================================================================

\begin{table}[H]
\centering
\resizebox{0.5\columnwidth}{!}{%
  \begin{tabular}{|c|c|c|c|}
    \hline
    \multicolumn{4}{|c|}{$\lambda = \SI{463}{\nm}$} \\ \hline
    \multicolumn{2}{|c|}{Messung 1} & \multicolumn{2}{c|}{Messung 2} \\ \hline
    $U_G {[\si{mV}]}$ & $U_{ph} {[\si{mV}]}$ & $U_G {[\si{mV}]}$ & $U_{ph} {[\si{mV}]}$ \\ \hline
    0,5   & 1107,0   & 0,5   & 1130,0   \\ \hline
    31,1  & 901,0    & 34,3  & 866,0    \\ \hline
    87,5  & 537,0    & 91,9  & 522,0    \\ \hline
    133,2 & 313,1    & 135,0 & 307,4    \\ \hline
    152,1 & 236,9    & 152,1 & 242,8    \\ \hline
    192,5 & 126,6    & 190,6 & 130,0    \\ \hline
    227,5 & 68,5     & 227,0 & 68,8     \\ \hline
    291,0 & 18,8     & 287,4 & 21,1     \\ \hline
    334,9 & 1,9      & 329,2 & 2,8      \\ \hline
    345,9 & 0,6      & 349,4 & 0,0      \\ \hline
  \end{tabular}%
}
\caption{Messwerte der Photospannung $U_{ph}$ bei Gegenspannung $U_G$ für $\lambda = \SI{463}{\nm}$, mit $\Delta U_{ph} = 0.1 \cdot U_{ph} + \SI{10}{\milli\volt}$ und $\Delta U_{G} = \SI{10}{\milli\volt}$.}
\label{tab:463nm}
\end{table}
%=======================================================================================================================================================================================================
\begin{table}[H]
\centering
\resizebox{\columnwidth}{!}{%
  \begin{tabular}{|c|c|c|c|c|c|c|c|}
    \hline
    \multicolumn{4}{|c|}{$\lambda = \SI{546}{\nm}$}
      & \multicolumn{4}{c|}{$\lambda = \SI{578}{\nm}$} \\ \hline
    \multicolumn{2}{|c|}{Messung 1} & \multicolumn{2}{c|}{Messung 2}
      & \multicolumn{2}{c|}{Messung 1} & \multicolumn{2}{c|}{Messung 2} \\ \hline
    $U_G {[\si{mV}]}$ & $U_{ph} {[\si{mV}]}$ & $U_G {[\si{mV}]}$ & $U_{ph} {[\si{mV}]}$
      & $U_G {[\si{mV}]}$ & $U_{ph} {[\si{mV}]}$ & $U_G {[\si{mV}]}$ & $U_{ph} {[\si{mV}]}$ \\ \hline
    0,5   & 5700,0   & 0,5   & 5390,0
      & 0,5   & 644,0    & 0,5   & 565,0    \\ \hline
    32,7  & 2155,0   & 29,0  & 2444,0
      & 10,7  & 406,0    & 10,2  & 406,0    \\ \hline
    12,4  & 3900,0   & 9,1   & 4230,0
      & 21,6  & 287,5    & 23,6  & 276,8    \\ \hline
    52,9  & 1165,0   & 51,0  & 1206,0
      & 31,1  & 217,2    & 29,6  & 225,1    \\ \hline
    69,9  & 677,0    & 61,7  & 874,0
      & 39,6  & 169,3    & 39,6  & 169,6    \\ \hline
    60,4  & 949,0    & 71,8  & 645,0
      & 54,9  & 117,2    & 50,5  & 122,7    \\ \hline
    103,2 & 212,9    & 101,0 & 242,1
      & 62,1  & 82,6     & 60,8  & 84,4     \\ \hline
    130,7 & 77,7     & 128,7 & 82,2
      & 73,3  & 55,7     & 69,6  & 63,7     \\ \hline
    150,8 & 13,8     & 150,8 & 18,3
      & 82,2  & 39,6     & 83,2  & 37,6     \\ \hline
    162,4 & 6,2      & 158,7 & 0,1
      & 119,8 & 0,0      & 116,6 & 2,4      \\ \hline
  \end{tabular}%
}
\caption{Messwerte der Photospannung $U_{ph}$ bei Gegenspannung $U_G$ für $\lambda = \SI{546}{\nm}$ und $\lambda = \SI{578}{\nm}$, wobei $\Delta U_{ph} = 0.1 \cdot U_{ph} + \SI{10}{\milli\volt}$ und $\Delta U_{G} = \SI{10}{\milli\volt}$.}
\label{tab:546and578}
\end{table}
%=======================================================================================================================================================================================================
\begin{table}[H]
\centering
\resizebox{0.5\columnwidth}{!} \\ \hline
    $U_G {[\si{mV}]}$ & $U_{ph} {[\si{mV}]}$ & $U_G {[\si{mV}]}$ & $U_{ph} {[\si{mV}]}$ \\ \hline
    0,4    & 9200,0 & 0,5   & 4040,0 \\ \hline
    100,2  & 5760,0 & 106,6 & 2417,0 \\ \hline
    200,8  & 2939,0 & 203,3 & 1296,0 \\ \hline
    259,3  & 1806,0 & 235,6 &  974,0 \\ \hline
    299,3  & 1172,0 & 255,8 &  806,0 \\ \hline
    330,4  &  823,0 & 279,8 &  630,0 \\ \hline
    364,5  &  549,0 & 303,6 &  491,0 \\ \hline
    402,0  &  337,6 & 350,8 &  282,4 \\ \hline
    453,0  &  111,6 & 404,0 &  147,8 \\ \hline
    504,0  &   12,2 & 450,0 &   58,8 \\ \hline
    1020,0 &    1,5 & 510,0 &   10,0 \\ \hline
  \end{tabular}%
}
\caption{Messwerte der Photospannung $U_{ph}$ bei Gegenspannung $U_G$ für $\lambda = \SI{365}{\nm}$ (Maximalwerte und 50\%-Punkt), wobei $\Delta U_{ph} = 0.1 \cdot U_{ph} + \SI{10}{\milli\volt}$ und $\Delta U_{G} = \SI{10}{\milli\volt}$.}
\label{tab:365_max_50}
\end{table}
%=======================================================================================================================================================================================================
\subsection*{Balmer-Serie}
\begin{table}[H]
\centering
\resizebox{\columnwidth}{!}{%
  \begin{tabular}{|c|c|c|c|c|c|c|c|}
    \hline
    \multicolumn{8}{|c|}{Spektrallinie Hg} \\ \hline
   % \multicolumn{8}{|c|}{1.\ Ordnung} \\ \hline
    $\omega_B$ [\si{\degree}] & $\Delta\omega_B$ [\si{\degree}] 
  & Spektrallinien & $\omega_G$ [\si{\degree}] & $\Delta\omega_G$ [\si{\degree}] 
  & Farbe            & Dicke $d$ [Skt] & $\Delta d$ [Skt] \\ \hline
    145,0 & 0,5 &        & 48,0  & 0,5 & violett       & 4  & 1 \\ \hline
    145,0 & 0,5 &        & 49,0  & 0,5 & violett       & 2  & 1 \\ \hline
    145,0 & 0,5 &        & 49,5  & 0,5 & violett       & 2  & 1 \\ \hline
    145,0 & 0,5 &        & 50,5  & 0,5 & violett/blau  & 3  & 1 \\ \hline
    145,0 & 0,5 &        & 51,0  & 0,5 & violett/blau  & 3  & 1 \\ \hline
    145,0 & 0,5 &        & 51,0  & 0,5 & blau          & 4  & 1 \\ \hline
    145,0 & 0,5 &        & 55,5  & 0,5 & türkis        & 1  & 0,1 \\ \hline
    145,0 & 0,5 &        & 61,0  & 0,5 & grün          & 3  & 1 \\ \hline
    145,0 & 0,5 &        & 64,0  & 0,5 & gelb          & 5  & 1 \\ \hline
    145,0 & 0,5 &        & 64,5  & 0,5 & gelb          & 5  & 1 \\ \hline
    145,0 & 0,5 &        & 69,0  & 0,5 & rot           & 5  & 1 \\ \hline
    135,0 & 0,5 &        & 68,0  & 0,5 & grün          & 8  & 1 \\ \hline
    135,0 & 0,5 &        & 71,0  & 0,5 & gelb          & 10 & 1 \\ \hline
    135,0 & 0,5 &        & 71,5  & 0,5 & gelb          & 11 & 1 \\ \hline
    155,0 & 0,5 &        & 61,0  & 0,5 & rot           & 1  & 0,1 \\ \hline
    155,0 & 0,5 &        & 61,5  & 0,5 & rot           & 1  & 0,1 \\ \hline
    155,0 & 0,5 &        & 62,5  & 0,5 & rot           & 2  & 1 \\ \hline
  \end{tabular}%
}
\caption{Spektrallinien der Hg-Dampflampe 1.\ Ordnung, gemessen an den Winkelpositionen und beobachteter Farbe. Hierbei ist $d$ die Dicke der Spektrallinien (in Strichpunkten), 
$\omega_B$ der Winkel der optischen Bank und $\omega_G$ der Winkel des Gitters. } 
\label{tab:Hg_lines}
\end{table}
%=======================================================================================================================================================================================================
\begin{table}[H]
\centering
\resizebox{\columnwidth}{!}{%
  \begin{tabular}{|c|c|c|c|c|c|c|c|}
    \hline
    \multicolumn{8}{|c|}{Spektrallinie H/Deuterium} \\ \hline
    %\multicolumn{8}{|c|}{1.\ Ordnung} \\ \hline
    $\omega_B$ [\si{\degree}] & $\Delta\omega_B$ [\si{\degree}]
  & Spektrallinien & $\omega_G$ [\si{\degree}] & $\Delta\omega_G$ [\si{\degree}]
  & Farbe            & $d$ [Skt] & $\Delta d$ [Skt] \\ \hline
    145,0 & 0,5 & & 51,0 & 0,5 & violett & 3 & 1 \\ \hline
    145,0 & 0,5 & & 55,5 & 0,5 & türkis  & 1 & 0,1 \\ \hline
    155,0 & 0,5 & & 61,5 & 0,5 & rot     & 1 & 0,1 \\ \hline
  \end{tabular}%
}
\caption{Spektrallinien der H/Deuterium-Lampe in erster Ordnung. Hierbei ist $d$ die Dicke der Spektrallinien (in Strichpunkten), $\omega_B$ der Winkel der Blende und $\omega_G$ der Beugungswinkel.}
\label{tab:H_D_lines}
\end{table}
