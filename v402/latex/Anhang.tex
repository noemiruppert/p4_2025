\chapter{Anhang}
\section{Abbildungen}
\subsection*{Photoeffekt}

\section{Tabellen}
\subsection*{Photoeffekt}
%=======================================================================================================================================================================================================
\begin{table}[H]
\centering
\resizebox{\columnwidth}{!}{
\begin{tabular}{|c|c|c|c|c|c|c|c|}
\hline
\multicolumn{4}{|c|}{$\lambda = \SI{365}{\nm}$}
  & \multicolumn{4}{c|}{$\lambda = \SI{405}{\nm}$} \\ \hline
\multicolumn{2}{|c|}{Messung 1} & \multicolumn{2}{c|}{Messung 2}
  & \multicolumn{2}{c|}{Messung 1} & \multicolumn{2}{c|}{Messung 2} \\ \hline
$U_G {[\si{mV}]}$   & $U_{ph} {[\si{mV}]}$  & $U_G {[\si{mV}]}$  & $U_{ph} {[\si{mV}]}$ & $U_G {[\si{mV}]}$    & $U_{ph} {[\si{mV}]}$ & $U_G {[\si{mV}]}$  & $U_{ph} {[\si{mV}]}$ \\ \hline
0,5             & 2380,0            & 0,5           & 2382,0           & 0,5             & 900,0            & 0,5           & 914,0            \\ \hline
30,5            & 2070,0            & 35,6          & 2065,0           & 41,5            & 722,0            & 36,0            & 740,0            \\ \hline
84,2            & 1630,0            & 84,6          & 1624,0           & 49,4            & 680,0            & 53,9          & 665,0            \\ \hline
121,0             & 1325,0            & 120,5         & 1334,0           & 105,4           & 450,0            & 102,0           & 464,0            \\ \hline
152,9           & 1071,0            & 156,3         & 1061,0           & 149,6           & 293,8          & 152,9         & 286,9,0          \\ \hline
160,7           & 1025,0            & 170,3         & 951,0            & 167,7           & 245,0            & 165,6         & 250,5          \\ \hline
182,4           & 883,0             & 180,8         & 892,0            & 174,5           & 226,9          & 178,2         & 215,6          \\ \hline
190,7           & 823,0             & 194,8         & 795,0            & 201,7           & 161,2          & 200,3         & 164,8          \\ \hline
217,9           & 655,0             & 219,4         & 653,0            & 213,7           & 139,5          & 217,5         & 230,0            \\ \hline
270,0             & 396,0             & 271,1         & 392,0            & 247,4           & 86,1           & 254,1         & 78,8           \\ \hline
287,0             & 333,3           & 287,4         & 325,0            & 267,7           & 63,5           & 273,2         & 58,2           \\ \hline
355,7           & 152,7           & 359,2         & 148,8          & 289,6           & 45,8           & 293,0           & 24,8           \\ \hline
396,3           & 79,9            & 397,2         & 95,6           & 301,9           & 38,8           & 304,7         & 36,5           \\ \hline
452,0             & 32,9            & 455,0           & 29,2           & 352,9           & 14,8           & 345,3         & 17,6           \\ \hline
472,0             & 14,1            & 477,0          & 11,7           & 378,1           & 5,1            & 384,8         & 3,2            \\ \hline
517,0             & 1,1             & 516,0           & 1,4            & 410,0             & 1,1            & 405,0           & 1,1            \\ \hline
\end{tabular}%
}
\caption{Messwerte der Photospannung $U_{ph}$ bei Gegenspannung $U_G$ für $\lambda$ = $\SI{365}{\nm}$ und $\lambda$ = $\SI{405}{\nm}$, wobei $\Delta U_{ph} = 0.1 \cdot U_{ph} + 10\,\si{\milli\volt}, \quad
 \Delta U_{G} = 10 \si{\milli\volt}$}
\label{tab:365and405}
\end{table}
%=======================================================================================================================================================================================================

\begin{table}[H]
\centering
\resizebox{0.5\columnwidth}{!}{%
  \begin{tabular}{|c|c|c|c|}
    \hline
    \multicolumn{4}{|c|}{$\lambda = \SI{463}{\nm}$} \\ \hline
    \multicolumn{2}{|c|}{Messung 1} & \multicolumn{2}{c|}{Messung 2} \\ \hline
    $U_G {[\si{mV}]}$ & $U_{ph} {[\si{mV}]}$ & $U_G {[\si{mV}]}$ & $U_{ph} {[\si{mV}]}$ \\ \hline
    0,5   & 1107,0   & 0,5   & 1130,0   \\ \hline
    31,1  & 901,0    & 34,3  & 866,0    \\ \hline
    87,5  & 537,0    & 91,9  & 522,0    \\ \hline
    133,2 & 313,1    & 135,0 & 307,4    \\ \hline
    152,1 & 236,9    & 152,1 & 242,8    \\ \hline
    192,5 & 126,6    & 190,6 & 130,0    \\ \hline
    227,5 & 68,5     & 227,0 & 68,8     \\ \hline
    291,0 & 18,8     & 287,4 & 21,1     \\ \hline
    334,9 & 1,9      & 329,2 & 2,8      \\ \hline
    345,9 & 0,6      & 349,4 & 0,0      \\ \hline
  \end{tabular}%
}
\caption{Messwerte der Photospannung $U_{ph}$ bei Gegenspannung $U_G$ für $\lambda = \SI{463}{\nm}$, mit $\Delta U_{ph} = 0.1 \cdot U_{ph} + \SI{10}{\milli\volt}$ und $\Delta U_{G} = \SI{10}{\milli\volt}$.}
\label{tab:463nm}
\end{table}
%=======================================================================================================================================================================================================
\begin{table}[H]
\centering
\resizebox{\columnwidth}{!}{%
  \begin{tabular}{|c|c|c|c|c|c|c|c|}
    \hline
    \multicolumn{4}{|c|}{$\lambda = \SI{546}{\nm}$}
      & \multicolumn{4}{c|}{$\lambda = \SI{578}{\nm}$} \\ \hline
    \multicolumn{2}{|c|}{Messung 1} & \multicolumn{2}{c|}{Messung 2}
      & \multicolumn{2}{c|}{Messung 1} & \multicolumn{2}{c|}{Messung 2} \\ \hline
    $U_G {[\si{mV}]}$ & $U_{ph} {[\si{mV}]}$ & $U_G {[\si{mV}]}$ & $U_{ph} {[\si{mV}]}$
      & $U_G {[\si{mV}]}$ & $U_{ph} {[\si{mV}]}$ & $U_G {[\si{mV}]}$ & $U_{ph} {[\si{mV}]}$ \\ \hline
    0,5   & 5700,0   & 0,5   & 5390,0
      & 0,5   & 644,0    & 0,5   & 565,0    \\ \hline
    32,7  & 2155,0   & 29,0  & 2444,0
      & 10,7  & 406,0    & 10,2  & 406,0    \\ \hline
    12,4  & 3900,0   & 9,1   & 4230,0
      & 21,6  & 287,5    & 23,6  & 276,8    \\ \hline
    52,9  & 1165,0   & 51,0  & 1206,0
      & 31,1  & 217,2    & 29,6  & 225,1    \\ \hline
    69,9  & 677,0    & 61,7  & 874,0
      & 39,6  & 169,3    & 39,6  & 169,6    \\ \hline
    60,4  & 949,0    & 71,8  & 645,0
      & 54,9  & 117,2    & 50,5  & 122,7    \\ \hline
    103,2 & 212,9    & 101,0 & 242,1
      & 62,1  & 82,6     & 60,8  & 84,4     \\ \hline
    130,7 & 77,7     & 128,7 & 82,2
      & 73,3  & 55,7     & 69,6  & 63,7     \\ \hline
    150,8 & 13,8     & 150,8 & 18,3
      & 82,2  & 39,6     & 83,2  & 37,6     \\ \hline
    162,4 & 6,2      & 158,7 & 0,1
      & 119,8 & 0,0      & 116,6 & 2,4      \\ \hline
  \end{tabular}%
}
\caption{Messwerte der Photospannung $U_{ph}$ bei Gegenspannung $U_G$ für $\lambda = \SI{546}{\nm}$ und $\lambda = \SI{578}{\nm}$, wobei $\Delta U_{ph} = 0.1 \cdot U_{ph} + \SI{10}{\milli\volt}$ und $\Delta U_{G} = \SI{10}{\milli\volt}$.}
\label{tab:546and578}
\end{table}
%=======================================================================================================================================================================================================
\begin{table}[H]
\centering
\resizebox{0.5\columnwidth}{!} \\ \hline
    $U_G {[\si{mV}]}$ & $U_{ph} {[\si{mV}]}$ & $U_G {[\si{mV}]}$ & $U_{ph} {[\si{mV}]}$ \\ \hline
    0,4    & 9200,0 & 0,5   & 4040,0 \\ \hline
    100,2  & 5760,0 & 106,6 & 2417,0 \\ \hline
    200,8  & 2939,0 & 203,3 & 1296,0 \\ \hline
    259,3  & 1806,0 & 235,6 &  974,0 \\ \hline
    299,3  & 1172,0 & 255,8 &  806,0 \\ \hline
    330,4  &  823,0 & 279,8 &  630,0 \\ \hline
    364,5  &  549,0 & 303,6 &  491,0 \\ \hline
    402,0  &  337,6 & 350,8 &  282,4 \\ \hline
    453,0  &  111,6 & 404,0 &  147,8 \\ \hline
    504,0  &   12,2 & 450,0 &   58,8 \\ \hline
    1020,0 &    1,5 & 510,0 &   10,0 \\ \hline
  \end{tabular}%
}
\caption{Messwerte der Photospannung $U_{ph}$ bei Gegenspannung $U_G$ für $\lambda = \SI{365}{\nm}$ (Maximalwerte und 50\%-Punkt), wobei $\Delta U_{ph} = 0.1 \cdot U_{ph} + \SI{10}{\milli\volt}$ und $\Delta U_{G} = \SI{10}{\milli\volt}$.}
\label{tab:365_max_50}
\end{table}
%=======================================================================================================================================================================================================
\begin{table}[H]
\centering
\resizebox{0.75\columnwidth}{!}{%
\begin{tabular}{|c|c|c|c|c|c|}
\hline
$U$ [\si{\milli\volt}] & $\Delta U$ [\si{\milli\volt}] 
  & $I$ [\si{\pico\ampere}] & $\Delta I$ [\si{\pico\ampere}] 
  & $\sqrt{I - I_0}$ [\si{\sqrt{\pico\ampere}}] 
  & $\Delta\sqrt{I - I_0}$ [\si{\sqrt{\pico\ampere}}] \\ \hline
   2.15 &  43.00 & 2380.00 & 248.00 &  48.77 & 2.54 \\ \hline
 131.15 &  43.00 & 2070.00 & 217.00 &  45.49 & 2.39 \\ \hline
 362.06 &  43.00 & 1630.00 & 173.00 &  40.36 & 2.14 \\ \hline
 520.30 &  43.00 & 1325.00 & 142.50 & 36.39 & 1.96 \\ \hline
 657.47 &  43.00 & 1071.00 & 117.10 & 32.71 & 1.79 \\ \hline
 691.01 &  43.00 & 1025.00 & 112.50 & 32.00 & 1.76 \\ \hline
 784.32 &  43.00 &  883.00 &  98.30 & 29.70 & 1.66 \\ \hline
 820.01 &  43.00 &  823.00 &  92.30 & 28.67 & 1.61 \\ \hline
 936.97 &  43.00 &  655.00 &  75.50 & 25.57 & 1.48 \\ \hline
1161.00 &  43.00 &  396.00 &  49.60 & 19.87 & 1.25 \\ \hline
1234.10 &  43.00 &  333.30 &  43.33 & 18.23 & 1.19 \\ \hline
1529.51 &  43.00 &  152.70 &  25.27 & 12.31 & 1.03 \\ \hline
1704.09 &  43.00 &   79.90 &  17.99 &  8.88 & 1.01 \\ \hline
1943.60 &  43.00 &   32.90 &  13.29 &  5.64 & 1.18 \\ \hline
2029.60 &  43.00 &   14.10 &  11.41 &  3.61 & 1.58 \\ \hline
2223.10 &  43.00 &    1.10 &  10.11 &  0.00 & 0.00 \\ \hline
\end{tabular}%
}
\caption{Gemessenen Werte für die erste Messung bei $\lambda=\SI{365}{\nano\metre}$.  
Die Sättigungsspannung liegt bei $U_G= \SI{517.00}{\milli\volt}$, daraus folgt  
$I_0 = \SI{1.10 \pm 10.11}{\pico\ampere}$.}
\label{tab:365_first}
\end{table}

\begin{table}[H]
\centering
\resizebox{0.75\columnwidth}{!}{%
\begin{tabular}{|c|c|}
\hline
\textbf{Parameter} & \textbf{Wert} \\ \hline
Steigung $m$ [\si{\sqrt{\pico\ampere}/\milli\volt}]
  & $\num{-2.22e-2} \pm \num{5.58e-4}$ \\ \hline
Achsenabschnitt $b$ [\si{\sqrt{\pico\ampere}}]
  & $\num{4.70e1} \pm \num{7.48e-1}$ \\ \hline
$\mathrm{\chi}^2$
  & $\num{8.28}$ \\ \hline
Freiheitsgrade (dof)
  & 13 \\ \hline
$\mathrm{\chi}^2 / \mathrm{dof}$
  & $\num{0.637}$ \\ \hline
Abbrems­spannung $U_0$ [\si{\milli\volt}]
  & $\num{2117.12 \pm 62.98}$ \\ \hline
\end{tabular}%
}
\caption{Ergebnisse des gewichteten linearen $\mathrm{\chi}^2$-Fits zur Bestimmung der Abbrems­spannung für die erste Messung bei $\lambda=\SI{365}{\nano\metre}$. Die hier gezeigten Werte stammen aus Tabelle \ref{tab:365_first}.}
\label{tab:365_first_chi2}
\end{table}

%=======================================================================================================================================================================================================

\begin{table}[H]
\centering
\resizebox{0.75\columnwidth}{!}{%
\begin{tabular}{|c|c|c|c|c|c|}
\hline
$U$ [\si{\milli\volt}] & $\Delta U$ [\si{\milli\volt}] 
  & $I$ [\si{\pico\ampere}] & $\Delta I$ [\si{\pico\ampere}] 
  & $\sqrt{I - I_0}$ [\si{\sqrt{\pico\ampere}}] 
  & $\Delta\sqrt{I - I_0}$ [\si{\sqrt{\pico\ampere}}] \\ \hline
   2.15 &  43.00 & 2382.00 & 248.20 &  48.79 &  2.54 \\ \hline
 153.08 &  43.00 & 2065.00 & 216.50 &  45.43 &  2.38 \\ \hline
 363.78 &  43.00 & 1624.00 & 172.40 &  40.28 &  2.14 \\ \hline
 518.15 &  43.00 & 1334.00 & 143.40 &  36.50 &  1.96 \\ \hline
 672.09 &  43.00 & 1061.00 & 116.10 &  32.55 &  1.78 \\ \hline
 732.29 &  43.00 &  951.00 & 105.10 &  30.82 &  1.71 \\ \hline
 777.44 &  43.00 &  892.00 &  99.20 &  29.84 &  1.66 \\ \hline
 837.64 &  43.00 &  795.00 &  89.50 &  28.17 &  1.59 \\ \hline
 943.42 &  43.00 &  653.00 &  75.30 &  25.53 &  1.47 \\ \hline
1165.73 &  43.00 &  392.00 &  49.20 &  19.76 &  1.24 \\ \hline
1235.82 &  43.00 &  325.00 &  42.50 &  17.99 &  1.18 \\ \hline
1544.56 &  43.00 &  148.80 &  24.88 &  12.14 &  1.02 \\ \hline
1707.96 &  43.00 &   95.60 &  19.56 &   9.71 &  1.01 \\ \hline
1956.50 &  43.00 &   29.20 &  12.92 &   5.27 &  1.23 \\ \hline
2051.10 &  43.00 &   11.70 &  11.17 &   3.21 &  1.74 \\ \hline
2218.80 &  43.00 &    1.40 &  10.14 &   0.00 &  0.00 \\ \hline
\end{tabular}%
}
\caption{Gemessenen Werte für die zweite Messung bei $\lambda=\SI{365}{\nano\metre}$.  
Die Sättigungsspannung liegt bei $U_G=\SI{516.00}{\milli\volt}$, daraus folgt  
$I_0 = \SI[separate-uncertainty=true]{1.40 \pm 10.14}{\pico\ampere}$.}
\label{tab:365_second}
\end{table}

\begin{table}[H]
\centering
\resizebox{0.75\columnwidth}{!}{%
\begin{tabular}{|c|c|}
\hline
\textbf{Parameter} & \textbf{Wert} \\ \hline
Steigung $m$ [\si{\sqrt{\pico\ampere}/\milli\volt}] 
  & $\num{-2.20e-2} \pm \num{5.73e-4}$ \\ \hline
Achsenabschnitt $b$ [\si{\sqrt{\pico\ampere}}] 
  & $\num{4.69e1} \pm \num{7.63e-1}$ \\ \hline
$\mathrm{\chi}^2$ 
  & $\num{8.45}$ \\ \hline
Freiheitsgrade (dof) 
  & 13 \\ \hline
$\mathrm{\chi}^2 / \mathrm{dof}$ 
  & $\num{0.650}$ \\ \hline
Abbrems­spannung $U_0$ [\si{\milli\volt}] 
  & $\num{2131.82 \pm 65.47}$ \\ \hline
\end{tabular}%
}
\caption{Ergebnisse des gewichteten linearen $\mathrm{\chi}^2$-Fits zur Bestimmung der Abbrems­spannung für die zweite Messung bei $\lambda=\SI{365}{\nano\metre}$. Die hier gezeigten Werte stammen aus Tabelle \ref{tab:365_second}.}
\label{tab:365_second_chi2}
\end{table}


%=======================================================================================================================================================================================================
\begin{table}[H]
\centering
\resizebox{0.75\columnwidth}{!}{%
\begin{tabular}{|c|c|c|c|c|c|}
\hline
$U$ [\si{\milli\volt}] & $\Delta U$ [\si{\milli\volt}] 
  & $I$ [\si{\pico\ampere}] & $\Delta I$ [\si{\pico\ampere}] 
  & $\sqrt{I - I_0}$ [\si{\sqrt{\pico\ampere}}] 
  & $\Delta\sqrt{I - I_0}$ [\si{\sqrt{\pico\ampere}}] \\ \hline
   2.15 &  43.00 & 4040.00 & 414.00 &  63.48 & 3.26 \\ \hline
 458.38 &  43.00 & 2417.00 & 251.70 &  49.06 & 2.57 \\ \hline
 874.19 &  43.00 & 1296.00 & 139.60 &  35.86 & 1.95 \\ \hline
1013.08 &  43.00 &  974.00 & 107.40 &  31.05 & 1.73 \\ \hline
1099.94 &  43.00 &  806.00 &  90.60 &  28.21 & 1.61 \\ \hline
1203.14 &  43.00 &  630.00 &  73.00 &  24.90 & 1.47 \\ \hline
1305.48 &  43.00 &  491.00 &  59.10 &  21.93 & 1.35 \\ \hline
1508.44 &  43.00 &  282.40 &  38.24 &  16.50 & 1.16 \\ \hline
1737.20 &  43.00 &  147.80 &  24.78 &  11.74 & 1.06 \\ \hline
1935.00 &  43.00 &   58.80 &  15.88 &   6.99 & 1.14 \\ \hline
2193.00 &  43.00 &   10.00 &  11.00 &   0.00 & 0.00 \\ \hline
\end{tabular}%
}
\caption{Gemessenen Werte für die Messung bei 50 \% Intensität und $\lambda=\SI{365}{\nano\metre}$.  
Die Sättigungsspannung liegt bei $U_G=\SI{510.00}{\milli\volt}$, daraus folgt  
$I_0 = \SI[separate-uncertainty=true]{10.00 \pm 11.00}{\pico\ampere}$.}
\label{tab:365_50pct}
\end{table}

\begin{table}[H]
\centering
\resizebox{0.75\columnwidth}{!}{%
\begin{tabular}{|c|c|}
\hline
\textbf{Parameter} & \textbf{Wert} \\ \hline
Steigung $m$ [\si{\sqrt{\pico\ampere}/\milli\volt}] 
  & $\num{-2.79e-2} \pm \num{9.92e-4}$ \\ \hline
Achsenabschnitt $b$ [\si{\sqrt{\pico\ampere}}] 
  & $\num{5.96e1} \pm \num{1.45e0}$ \\ \hline
$\mathrm{\chi}^2$ 
  & $\num{6.55}$ \\ \hline
Freiheitsgrade (dof) 
  & 8 \\ \hline
$\mathrm{\chi}^2 / \mathrm{dof}$ 
  & $\num{0.819}$ \\ \hline
Abbrems­spannung $U_0$ [\si{\milli\volt}] 
  & $\num{2136.20 \pm 92.03}$ \\ \hline
\end{tabular}%
}
\caption{Ergebnisse des gewichteten linearen $\mathrm{\chi}^2$-Fits zur Bestimmung der Abbrems­spannung für die Messung bei $\lambda=\SI{365}{\nano\metre}$ mit 50\% Intensität. Die hier gezeigten Werte stammen aus Tabelle \ref{tab:365_50}.}
\label{tab:365_50_chi2}
\end{table}

%=======================================================================================================================================================================================================
\begin{table}[H]
\centering
\resizebox{0.75\columnwidth}{!}{%
\begin{tabular}{|c|c|c|c|c|c|}
\hline
$U$ [\si{\milli\volt}] & $\Delta U$ [\si{\milli\volt}] 
  & $I$ [\si{\pico\ampere}] & $\Delta I$ [\si{\pico\ampere}] 
  & $\sqrt{I - I_0}$ [\si{\sqrt{\pico\ampere}}] 
  & $\Delta\sqrt{I - I_0}$ [\si{\sqrt{\pico\ampere}}] \\ \hline
   1.72 &  43.00 & 9200.00 &  930.00 &  95.91 & 4.85 \\ \hline
 430.86 &  43.00 & 5760.00 &  586.00 &  75.88 & 3.86 \\ \hline
 863.44 &  43.00 & 2939.00 &  303.90 &  54.20 & 2.80 \\ \hline
1114.99 &  43.00 & 1806.00 &  190.60 &  42.48 & 2.24 \\ \hline
1286.99 &  43.00 & 1172.00 &  127.20 &  34.21 & 1.86 \\ \hline
1420.72 &  43.00 &  823.00 &   92.30 &  28.66 & 1.61 \\ \hline
1567.35 &  43.00 &  549.00 &   64.90 &  23.40 & 1.39 \\ \hline
1728.60 &  43.00 &  337.60 &   43.76 &  18.33 & 1.19 \\ \hline
1947.90 &  43.00 &  111.60 &   21.16 &  10.49 & 1.01 \\ \hline
2167.20 &  43.00 &   12.20 &   11.22 &   3.27 & 1.72 \\ \hline
4386.00 &  43.00 &    1.50 &   10.15 &   0.00 & 0.00 \\ \hline
\end{tabular}%
}
\caption{Gemessenen Werte für die Messung bei maximaler Intensität und $\lambda=\SI{365}{\nano\metre}$.  
Die Sättigungsspannung liegt bei $U_G=\SI{1020.00}{\milli\volt}$, daraus folgt  
$I_0 = \SI[separate-uncertainty=true]{1.50 \pm 10.15}{\pico\ampere}$.}
\label{tab:365_max}
\end{table}

\begin{table}[H]
\centering
\resizebox{0.75\columnwidth}{!}{%
\begin{tabular}{|c|c|}
\hline
\textbf{Parameter} & \textbf{Wert} \\ \hline
Steigung $m$ [\si{\sqrt{\pico\ampere}/\milli\volt}] 
  & $\num{-4.00e-2} \pm \num{1.57e-3}$ \\ \hline
Achsenabschnitt $b$ [\si{\sqrt{\pico\ampere}}] 
  & $\num{8.76e1} \pm \num{2.64e0}$ \\ \hline
$\mathrm{\chi}^2$ 
  & $\num{11.74}$ \\ \hline
Freiheitsgrade (dof) 
  & 8 \\ \hline
$\mathrm{\chi}^2 / \mathrm{dof}$ 
  & $\num{1.467}$ \\ \hline
Abbrems­spannung $U_0$ [\si{\milli\volt}] 
  & $\num{2190.00 \pm 108.37}$ \\ \hline
\end{tabular}%
}
\caption{Ergebnisse des gewichteten linearen $\mathrm{\chi}^2$-Fits zur Bestimmung der Abbrems­spannung für die Messung bei $\lambda=\SI{365}{\nano\metre}$ mit maximaler Intensität. Die hier gezeigten Werte stammen aus Tabelle \ref{tab:365_max}.}
\label{tab:365_max_chi2}
\end{table}

%=======================================================================================================================================================================================================
\begin{table}[H]
\centering
\resizebox{0.75\columnwidth}{!}{%
\begin{tabular}{|c|c|c|c|c|c|}
\hline
$U$ [\si{\milli\volt}] & $\Delta U$ [\si{\milli\volt}] 
  & $I$ [\si{\pico\ampere}] & $\Delta I$ [\si{\pico\ampere}] 
  & $\sqrt{I - I_0}$ [\si{\sqrt{\pico\ampere}}] 
  & $\Delta\sqrt{I - I_0}$ [\si{\sqrt{\pico\ampere}}] \\ \hline
   2.15 &  43.00 &  900.00 & 100.00 &  29.98 & 1.67 \\ \hline
 178.45 &  43.00 &  722.00 &  82.20 &  26.85 & 1.53 \\ \hline
 212.42 &  43.00 &  680.00 &  78.00 &  26.06 & 1.50 \\ \hline
 453.22 &  43.00 &  450.00 &  55.00 &  21.19 & 1.30 \\ \hline
 643.28 &  43.00 &  293.80 &  39.38 &  17.11 & 1.15 \\ \hline
 721.11 &  43.00 &  245.00 &  34.50 &  15.62 & 1.10 \\ \hline
 750.35 &  43.00 &  226.90 &  32.69 &  15.03 & 1.09 \\ \hline
 867.31 &  43.00 &  161.20 &  26.12 &  12.65 & 1.03 \\ \hline
 918.91 &  43.00 &  139.50 &  23.95 &  11.76 & 1.02 \\ \hline
1063.82 &  43.00 &   86.10 &  18.61 &   9.22 & 1.01 \\ \hline
1151.11 &  43.00 &   63.50 &  16.35 &   7.90 & 1.03 \\ \hline
1245.28 &  43.00 &   45.80 &  14.58 &   6.69 & 1.09 \\ \hline
1298.17 &  43.00 &   38.80 &  13.88 &   6.14 & 1.13 \\ \hline
1517.47 &  43.00 &   14.80 &  11.48 &   3.70 & 1.55 \\ \hline
1625.83 &  43.00 &    5.10 &  10.51 &   2.00 & 2.63 \\ \hline
1763.00 &  43.00 &    1.10 &  10.11 &   0.00 & 0.00 \\ \hline
\end{tabular}%
}
\caption{Gemessenen Werte für die erste Messung bei $\lambda=\SI{405}{\nano\metre}$.  
Die Sättigungsspannung liegt bei $U_G=\SI{410.00}{\milli\volt}$, daraus folgt  
$I_0 = \SI[separate-uncertainty=true]{1.10 \pm 10.11}{\pico\ampere}$.}
\label{tab:405_first}
\end{table}

\begin{table}[H]
\centering
\resizebox{0.75\columnwidth}{!}{%
\begin{tabular}{|c|c|}
\hline
\textbf{Parameter} & \textbf{Wert} \\ \hline
Steigung $m$ [\si{\sqrt{\pico\ampere}/\milli\volt}] 
  & $\num{-1.81e-2} \pm \num{5.50e-4}$ \\ \hline
Achsenabschnitt $b$ [\si{\sqrt{\pico\ampere}}] 
  & $\num{2.90e1} \pm \num{5.21e-1}$ \\ \hline
$\mathrm{\chi}^2$ 
  & $\num{5.80}$ \\ \hline
Freiheitsgrade (dof) 
  & 13 \\ \hline
$\mathrm{\chi}^2 / \mathrm{dof}$ 
  & $\num{0.446}$ \\ \hline
Abbrems­spannung $U_0$ [\si{\milli\volt}] 
  & $\num{1602.21 \pm 56.56}$ \\ \hline
\end{tabular}%
}
\caption{Ergebnisse des gewichteten linearen $\mathrm{\chi}^2$-Fits zur Bestimmung der Abbrems­spannung für die erste Messung bei $\lambda=\SI{405}{\nano\metre}$. Die hier gezeigten Werte stammen aus Tabelle \ref{tab:405_first}.}
\label{tab:405_first_chi2}
\end{table}

%=======================================================================================================================================================================================================
\begin{table}[H]
\centering
\resizebox{0.75\columnwidth}{!}{%
  \begin{tabular}{|c|c|c|c|c|c|}
    \hline
    $U$ [\si{\milli\volt}] & $\Delta U$ [\si{\milli\volt}]
      & $I$ [\si{\pico\ampere}] & $\Delta I$ [\si{\pico\ampere}]
      & $\sqrt{I - I_0}$ [\si{\sqrt{\pico\ampere}}]
      & $\Delta\sqrt{I - I_0}$ [\si{\sqrt{\pico\ampere}}] \\ \hline
       2.15 &  43.00 &  914.00 & 101.40 &  30.21 & 1.68 \\ \hline
 154.80 &  43.00 &  740.00 &  84.00 &  27.18 & 1.55 \\ \hline
 231.77 &  43.00 &  665.00 &  76.50 &  25.77 & 1.48 \\ \hline
 438.60 &  43.00 &  464.00 &  56.40 &  21.52 & 1.31 \\ \hline
 657.47 &  43.00 &  286.90 &  38.69 &  16.91 & 1.14 \\ \hline
 712.08 &  43.00 &  250.50 &  35.05 &  15.79 & 1.11 \\ \hline
 766.26 &  43.00 &  215.60 &  31.56 &  14.65 & 1.08 \\ \hline
 861.29 &  43.00 &  164.80 &  26.48 &  12.79 & 1.03 \\ \hline
 935.25 &  43.00 &  230.00 &  33.00 &  15.13 & 1.09 \\ \hline
1092.63 &  43.00 &   78.80 &  17.88 &   8.81 & 1.01 \\ \hline
1174.76 &  43.00 &   58.20 &  15.82 &   7.56 & 1.05 \\ \hline
1259.90 &  43.00 &   24.80 &  12.48 &   4.87 & 1.28 \\ \hline
1310.21 &  43.00 &   36.50 &  13.65 &   5.95 & 1.15 \\ \hline
1484.79 &  43.00 &   17.60 &  11.76 &   4.06 & 1.45 \\ \hline
1654.64 &  43.00 &    3.20 &  10.32 &   1.45 & 3.56 \\ \hline
1741.50 &  43.00 &    1.10 &  10.11 &   0.00 & 0.00 \\ \hline
  \end{tabular}%
}
\caption{Gemessenen Werte für die zweite Messung bei 
  $\lambda=\SI{405}{\nano\metre}$.  
  Die Sättigungsspannung liegt bei $U_G=\SI{405.00}{\milli\volt}$, daraus folgt  
  $I_0 = \SI[separate-uncertainty=true]{1.10 \pm 10.11}{\pico\ampere}$.}
\label{tab:405_second}
\end{table}

\begin{table}[H]
\centering
\resizebox{0.75\columnwidth}{!}{%
\begin{tabular}{|c|c|}
\hline
\textbf{Parameter} & \textbf{Wert} \\ \hline
Steigung $m$ [\si{\sqrt{\pico\ampere}/\milli\volt}] 
  & $\num{-1.83e-2} \pm \num{8.30e-4}$ \\ \hline
Achsenabschnitt $b$ [\si{\sqrt{\pico\ampere}}] 
  & $\num{2.95e1} \pm \num{7.84e-1}$ \\ \hline
$\mathrm{\chi}^2$ 
  & $\num{12.92}$ \\ \hline
Freiheitsgrade (dof) 
  & 13 \\ \hline
$\mathrm{\chi}^2 / \mathrm{dof}$ 
  & $\num{0.994}$ \\ \hline
Abbrems­spannung $U_0$ [\si{\milli\volt}] 
  & $\num{1612.02 \pm 84.74}$ \\ \hline
\end{tabular}%
}
\caption{Ergebnisse des gewichteten linearen $\mathrm{\chi}^2$-Fits zur Bestimmung der Abbrems­spannung für die zweite Messung bei $\lambda=\SI{405}{\nano\metre}$. Die hier gezeigten Werte stammen aus Tabelle \ref{tab:405_second}.}
\label{tab:405_second_chi2}
\end{table}

%=======================================================================================================================================================================================================
\begin{table}[H]
\centering
\resizebox{0.75\columnwidth}{!}{%
\begin{tabular}{|c|c|c|c|c|c|}
\hline
$U$ [\si{\milli\volt}] & $\Delta U$ [\si{\milli\volt}] 
  & $I$ [\si{\pico\ampere}] & $\Delta I$ [\si{\pico\ampere}] 
  & $\sqrt{I - I_0}$ [\si{\sqrt{\pico\ampere}}] 
  & $\Delta\sqrt{I - I_0}$ [\si{\sqrt{\pico\ampere}}] \\ \hline
   2.15 &  43.00 & 1107.00 & 120.70 &  33.26 & 1.81 \\ \hline
 133.73 &  43.00 &  901.00 & 100.10 &  30.01 & 1.67 \\ \hline
 376.25 &  43.00 &  537.00 &  63.70 &  23.16 & 1.38 \\ \hline
 572.76 &  43.00 &  313.10 &  41.31 &  17.68 & 1.17 \\ \hline
 654.03 &  43.00 &  236.90 &  33.69 &  15.37 & 1.10 \\ \hline
 827.75 &  43.00 &  126.60 &  22.66 &  11.22 & 1.01 \\ \hline
 978.25 &  43.00 &   68.50 &  16.85 &   8.24 & 1.02 \\ \hline
1251.30 &  43.00 &   18.80 &  11.88 &   4.27 & 1.39 \\ \hline
1440.07 &  43.00 &    1.90 &  10.19 &   1.14 & 4.47 \\ \hline
1487.37 &  43.00 &    0.60 &  10.06 &   0.00 & 0.00 \\ \hline
\end{tabular}%
}
\caption{Gemessenen Werte für die erste Messung bei $\lambda=\SI{463}{\nano\metre}$.  
Die Sättigungsspannung liegt bei $U_G=\SI{1487.37}{\milli\volt}$, daraus folgt  
$I_0 = \SI[separate-uncertainty=true]{0.60 \pm 10.06}{\pico\ampere}$.}
\label{tab:463_first}
\end{table}


\begin{table}[H]
\centering
\resizebox{0.75\columnwidth}{!}{%
\begin{tabular}{|c|c|}
\hline
\textbf{Parameter} & \textbf{Wert} \\ \hline
Steigung $m$ [\si{\sqrt{\pico\ampere}/\milli\volt}] 
  & $\num{-2.38e-2} \pm \num{1.22e-3}$ \\ \hline
Achsenabschnitt $b$ [\si{\sqrt{\pico\ampere}}] 
  & $\num{3.18e1} \pm \num{9.47e-1}$ \\ \hline
$\mathrm{\chi}^2$ 
  & $\num{6.31}$ \\ \hline
Freiheitsgrade (dof) 
  & 7 \\ \hline
$\mathrm{\chi}^2 / \mathrm{dof}$ 
  & $\num{0.901}$ \\ \hline
Abbrems­spannung $U_0$ [\si{\milli\volt}] 
  & $\num{1336.13 \pm 79.21}$ \\ \hline
\end{tabular}%
}
\caption{Ergebnisse des gewichteten linearen $\mathrm{\chi}^2$-Fits zur Bestimmung der Abbrems­spannung für die erste Messung bei $\lambda=\SI{463}{\nano\metre}$. Die hier gezeigten Werte stammen aus Tabelle \ref{tab:463_first}.}
\label{tab:463_first_chi2}
\end{table}

%=======================================================================================================================================================================================================
\begin{table}[H]
\centering
\resizebox{0.75\columnwidth}{!}{%
\begin{tabular}{|c|c|c|c|c|c|}
\hline
$U$ [\si{\milli\volt}] & $\Delta U$ [\si{\milli\volt}] 
  & $I$ [\si{\pico\ampere}] & $\Delta I$ [\si{\pico\ampere}] 
  & $\sqrt{I - I_0}$ [\si{\sqrt{\pico\ampere}}] 
  & $\Delta\sqrt{I - I_0}$ [\si{\sqrt{\pico\ampere}}] \\ \hline
   2.15 &  43.00 & 1130.00 & 123.00 &  33.62 & 1.83 \\ \hline
 147.49 &  43.00 &  866.00 &  96.60 &  29.43 & 1.64 \\ \hline
 395.17 &  43.00 &  522.00 &  62.20 &  22.85 & 1.36 \\ \hline
 580.50 &  43.00 &  307.40 &  40.74 &  17.53 & 1.16 \\ \hline
 654.03 &  43.00 &  242.80 &  34.28 &  15.58 & 1.10 \\ \hline
 819.58 &  43.00 &  130.00 &  23.00 &  11.40 & 1.01 \\ \hline
 976.10 &  43.00 &   68.80 &  16.88 &   8.29 & 1.02 \\ \hline
1235.82 &  43.00 &   21.10 &  12.11 &   4.59 & 1.32 \\ \hline
1415.56 &  43.00 &    2.80 &  10.28 &   1.67 & 3.08 \\ \hline
1502.42 &  43.00 &    0.01 &  10.00 &   0.00 & 0.00 \\ \hline
\end{tabular}%
}
\caption{Gemessenen Werte für die zweite Messung bei $\lambda=\SI{463}{\nano\metre}$.  
Die Sättigungsspannung liegt bei $U_G=\SI{1502.42}{\milli\volt}$, daraus folgt  
$I_0 = \SI[separate-uncertainty=true]{0.01 \pm 10.00}{\pico\ampere}$.}
\label{tab:463_second}
\end{table}

\begin{table}[H]
\centering
\resizebox{0.75\columnwidth}{!}{%
\begin{tabular}{|c|c|}
\hline
\textbf{Parameter} & \textbf{Wert} \\ \hline
Steigung $m$ [\si{\sqrt{\pico\ampere}/\milli\volt}] 
  & $\num{-2.36e-2} \pm \num{1.26e-3}$ \\ \hline
Achsenabschnitt $b$ [\si{\sqrt{\pico\ampere}}] 
  & $\num{3.18e1} \pm \num{9.94e-1}$ \\ \hline
$\mathrm{\chi}^2$ 
  & $\num{6.97}$ \\ \hline
Freiheitsgrade (dof) 
  & 7 \\ \hline
$\mathrm{\chi}^2 / \mathrm{dof}$ 
  & $\num{0.995}$ \\ \hline
Abbrems­spannung $U_0$ [\si{\milli\volt}] 
  & $\num{1347.46 \pm 83.36}$ \\ \hline
\end{tabular}%
}
\caption{Ergebnisse des gewichteten linearen $\mathrm{\chi}^2$-Fits zur Bestimmung der Abbrems­spannung für die zweite Messung bei $\lambda=\SI{463}{\nano\metre}$. Die hier gezeigten Werte stammen aus Tabelle \ref{tab:463_second}.}
\label{tab:463_second_chi2}
\end{table}

%=======================================================================================================================================================================================================
\begin{table}[H]
\centering
\resizebox{0.75\columnwidth}{!}{%
\begin{tabular}{|c|c|c|c|c|c|}
\hline
$U$ [\si{\milli\volt}] & $\Delta U$ [\si{\milli\volt}] 
  & $I$ [\si{\pico\ampere}] & $\Delta I$ [\si{\pico\ampere}] 
  & $\sqrt{I - I_0}$ [\si{\sqrt{\pico\ampere}}] 
  & $\Delta\sqrt{I - I_0}$ [\si{\sqrt{\pico\ampere}}] \\ \hline
   2.15 &  43.00 &   57.00 &   5.80 &   7.55 & 0.38 \\ \hline
 140.61 &  43.00 &   21.55 &   2.25 &   4.64 & 0.24 \\ \hline
  53.32 &  43.00 &   39.00 &   4.00 &   6.24 & 0.32 \\ \hline
 227.47 &  43.00 &   11.65 &   1.26 &   3.40 & 0.19 \\ \hline
 300.57 &  43.00 &    6.77 &   0.78 &   2.59 & 0.15 \\ \hline
 259.72 &  43.00 &    9.49 &   1.05 &   3.07 & 0.17 \\ \hline
 443.76 &  43.00 &    2.13 &   0.31 &   1.44 & 0.11 \\ \hline
 562.01 &  43.00 &    0.78 &   0.18 &   0.85 & 0.11 \\ \hline
 648.44 &  43.00 &    0.14 &   0.11 &   0.28 & 0.21 \\ \hline
 698.32 &  43.00 &    0.06 &   0.11 &   0.00 & 0.00 \\ \hline
\end{tabular}%
}
\caption{Gemessenen Werte für die erste Messung bei $\lambda=\SI{546}{\nano\metre}$.  
Die Sättigungsspannung liegt bei $U_G=\SI{698.32}{\milli\volt}$, daraus folgt  
$I_0 = \SI[separate-uncertainty=true]{0.06 \pm 0.11}{\pico\ampere}$.}
\label{tab:546_first}
\end{table}

\begin{table}[H]
\centering
\resizebox{0.75\columnwidth}{!}{%
\begin{tabular}{|c|c|}
\hline
\textbf{Parameter} & \textbf{Wert} \\ \hline
Steigung $m$ [\si{\sqrt{\pico\ampere}/\milli\volt}] 
  & $\num{-9.06e-3} \pm \num{9.59e-4}$ \\ \hline
Achsenabschnitt $b$ [\si{\sqrt{\pico\ampere}}] 
  & $\num{5.71e0} \pm \num{4.14e-1}$ \\ \hline
$\mathrm{\chi}^2$ 
  & $\num{60.11}$ \\ \hline
Freiheitsgrade (dof) 
  & 7 \\ \hline
$\mathrm{\chi}^2 / \mathrm{dof}$ 
  & $\num{8.587}$ \\ \hline
Abbrems­spannung $U_0$ [\si{\milli\volt}] 
  & $\num{630.24 \pm 80.86}$ \\ \hline
\end{tabular}%
}
\caption{Ergebnisse des gewichteten linearen $\mathrm{\chi}^2$-Fits zur Bestimmung der Abbrems­spannung für die erste Messung bei $\lambda=\SI{546}{\nano\metre}$. Die hier gezeigten Werte stammen aus Tabelle \ref{tab:546_first}.}
\label{tab:546_first_chi2}
\end{table}


%=======================================================================================================================================================================================================
\begin{table}[H]
\centering
\resizebox{0.75\columnwidth}{!}{%
\begin{tabular}{|c|c|c|c|c|c|}
\hline
$U$ [\si{\milli\volt}] & $\Delta U$ [\si{\milli\volt}] 
  & $I$ [\si{\pico\ampere}] & $\Delta I$ [\si{\pico\ampere}] 
  & $\sqrt{I - I_0}$ [\si{\sqrt{\pico\ampere}}] 
  & $\Delta\sqrt{I - I_0}$ [\si{\sqrt{\pico\ampere}}] \\ \hline
   2.15 &  43.00 &   53.90 &   5.49 &   7.34 & 0.37 \\ \hline
 124.70 &  43.00 &   24.44 &   2.54 &   4.94 & 0.26 \\ \hline
  39.13 &  43.00 &   42.30 &   4.33 &   6.50 & 0.33 \\ \hline
 219.30 &  43.00 &   12.06 &   1.31 &   3.47 & 0.19 \\ \hline
 265.31 &  43.00 &    8.74 &   0.97 &   2.96 & 0.16 \\ \hline
 308.74 &  43.00 &    6.45 &   0.74 &   2.54 & 0.15 \\ \hline
 434.30 &  43.00 &    2.42 &   0.34 &   1.56 & 0.11 \\ \hline
 553.41 &  43.00 &    0.82 &   0.18 &   0.91 & 0.10 \\ \hline
 648.44 &  43.00 &    0.18 &   0.12 &   0.43 & 0.14 \\ \hline
 682.41 &  43.00 &    0.00 &   0.10 &   0.00 & 0.00 \\ \hline
\end{tabular}%
}
\caption{Gemessenen Werte für die zweite Messung bei $\lambda=\SI{546}{\nano\metre}$.  
Die Sättigungsspannung liegt bei $U_G=\SI{682.41}{\milli\volt}$, daraus folgt  
$I_0 = \SI[separate-uncertainty=true]{0.00 \pm 0.10}{\pico\ampere}$.}
\label{tab:546_second}
\end{table}

\begin{table}[H]
\centering
\resizebox{0.75\columnwidth}{!}{%
\begin{tabular}{|c|c|}
\hline
\textbf{Parameter} & \textbf{Wert} \\ \hline
Steigung $m$ [\si{\sqrt{\pico\ampere}/\milli\volt}] 
  & $\num{-8.65e-3} \pm \num{9.51e-4}$ \\ \hline
Achsenabschnitt $b$ [\si{\sqrt{\pico\ampere}}] 
  & $\num{5.60e0} \pm \num{4.28e-1}$ \\ \hline
$\mathrm{\chi}^2$ 
  & $\num{69.41}$ \\ \hline
Freiheitsgrade (dof) 
  & 7 \\ \hline
$\mathrm{\chi}^2 / \mathrm{dof}$ 
  & $\num{9.915}$ \\ \hline
Abbrems­spannung $U_0$ [\si{\milli\volt}] 
  & $\num{647.40 \pm 86.69}$ \\ \hline
\end{tabular}%
}
\caption{Ergebnisse des gewichteten linearen $\mathrm{\chi}^2$-Fits zur Bestimmung der Abbrems­spannung für die zweite Messung bei $\lambda=\SI{546}{\nano\metre}$. Die hier gezeigten Werte stammen aus Tabelle \ref{tab:546_second}.}
\label{tab:546_second_chi2}
\end{table}


%=======================================================================================================================================================================================================
\begin{table}[H]
\centering
\resizebox{0.75\columnwidth}{!}{%
\begin{tabular}{|c|c|c|c|c|c|}
\hline
$U$ [\si{\milli\volt}] & $\Delta U$ [\si{\milli\volt}] 
  & $I$ [\si{\pico\ampere}] & $\Delta I$ [\si{\pico\ampere}] 
  & $\sqrt{I - I_0}$ [\si{\sqrt{\pico\ampere}}] 
  & $\Delta\sqrt{I - I_0}$ [\si{\sqrt{\pico\ampere}}] \\ \hline
   2.15 &  43.00 &   6.44 &   0.74 &   2.54 & 0.15 \\ \hline
  46.01 &  43.00 &   4.06 &   0.51 &   2.01 & 0.13 \\ \hline
  92.88 &  43.00 &   2.88 &   0.39 &   1.70 & 0.11 \\ \hline
 133.73 &  43.00 &   2.17 &   0.32 &   1.47 & 0.11 \\ \hline
 170.28 &  43.00 &   1.69 &   0.27 &   1.30 & 0.10 \\ \hline
 236.07 &  43.00 &   1.17 &   0.22 &   1.08 & 0.10 \\ \hline
 267.03 &  43.00 &   0.83 &   0.18 &   0.91 & 0.10 \\ \hline
 315.19 &  43.00 &   0.56 &   0.16 &   0.75 & 0.10 \\ \hline
 353.46 &  43.00 &   0.40 &   0.14 &   0.63 & 0.11 \\ \hline
 515.14 &  43.00 &   0.00 &   0.10 &   0.00 & 0.00 \\ \hline
\end{tabular}%
}
\caption{Gemessenen Werte für die erste Messung bei $\lambda=\SI{578}{\nano\metre}$.  
Die Sättigungsspannung liegt bei $U_G=\SI{515.14}{\milli\volt}$, daraus folgt  
$I_0 = \SI[separate-uncertainty=true]{0.00 \pm 0.10}{\pico\ampere}$.}
\label{tab:578_first}
\end{table}


\begin{table}[H]
\centering
\resizebox{0.75\columnwidth}{!}{%
\begin{tabular}{|c|c|}
\hline
\textbf{Parameter} & \textbf{Wert} \\ \hline
Steigung $m$ [\si{\sqrt{\pico\ampere}/\milli\volt}] 
  & $\num{-4.81e-3} \pm \num{3.82e-4}$ \\ \hline
Achsenabschnitt $b$ [\si{\sqrt{\pico\ampere}}] 
  & $\num{2.22e0} \pm \num{8.54e-2}$ \\ \hline
$\mathrm{\chi}^2$ 
  & $\num{8.54}$ \\ \hline
Freiheitsgrade (dof) 
  & 7 \\ \hline
$\mathrm{\chi}^2 / \mathrm{dof}$ 
  & $\num{1.219}$ \\ \hline
Abbrems­spannung $U_0$ [\si{\milli\volt}] 
  & $\num{461.54 \pm 40.73}$ \\ \hline
\end{tabular}%
}
\caption{Ergebnisse des gewichteten linearen $\mathrm{\chi}^2$-Fits zur Bestimmung der Abbremsspannung für die erste Messung bei $\lambda=\SI{578}{\nano\metre}$. Die hier gezeigten Werte stammen aus Tabelle \ref{tab:578_first}.}
\label{tab:578_first_chi2}
\end{table}


%=======================================================================================================================================================================================================
\begin{table}[H]
\centering
\resizebox{0.75\columnwidth}{!}{%
\begin{tabular}{|c|c|c|c|c|c|}
\hline
$U$ [\si{\milli\volt}] & $\Delta U$ [\si{\milli\volt}] 
  & $I$ [\si{\pico\ampere}] & $\Delta I$ [\si{\pico\ampere}] 
  & $\sqrt{I - I_0}$ [\si{\sqrt{\pico\ampere}}] 
  & $\Delta\sqrt{I - I_0}$ [\si{\sqrt{\pico\ampere}}] \\ \hline
   2.15 &  43.00 &   5.65 &   0.67 &   2.37 & 0.14 \\ \hline
  43.86 &  43.00 &   4.06 &   0.51 &   2.01 & 0.13 \\ \hline
 101.48 &  43.00 &   2.77 &   0.38 &   1.66 & 0.11 \\ \hline
 127.28 &  43.00 &   2.25 &   0.33 &   1.49 & 0.11 \\ \hline
 170.28 &  43.00 &   1.70 &   0.27 &   1.29 & 0.10 \\ \hline
 217.15 &  43.00 &   1.23 &   0.22 &   1.10 & 0.10 \\ \hline
 261.44 &  43.00 &   0.84 &   0.18 &   0.91 & 0.10 \\ \hline
 299.28 &  43.00 &   0.64 &   0.16 &   0.78 & 0.10 \\ \hline
 357.76 &  43.00 &   0.38 &   0.14 &   0.59 & 0.12 \\ \hline
 501.38 &  43.00 &   0.02 &   0.10 &   0.00 & 0.00 \\ \hline
\end{tabular}%
}
\caption{Gemessenen Werte für die zweite Messung bei $\lambda=\SI{578}{\nano\metre}$.  
Die Sättigungsspannung liegt bei $U_G=\SI{501.38}{\milli\volt}$, daraus folgt  
$I_0 = \SI[separate-uncertainty=true]{0.02 \pm 0.10}{\pico\ampere}$.}
\label{tab:578_second}
\end{table}


\begin{table}[H]
\centering
\resizebox{0.6\columnwidth}{!}{%
\begin{tabular}{|c|c|}
\hline
\textbf{Parameter} & \textbf{Wert} \\ \hline
Steigung $m$ [\si{\sqrt{\pico\ampere}/\milli\volt}] 
  & $\num{-4.77e-3} \pm \num{2.95e-4}$ \\ \hline
Achsenabschnitt $b$ [\si{\sqrt{\pico\ampere}}] 
  & $\num{2.18e0} \pm \num{6.37e-2}$ \\ \hline
$\mathrm{\chi}^2$ 
  & $\num{4.80}$ \\ \hline
Freiheitsgrade (dof) 
  & 7 \\ \hline
$\mathrm{\chi}^2 / \mathrm{dof}$ 
  & $\num{0.685}$ \\ \hline
Abbrems­spannung $U_0$ [\si{\milli\volt}] 
  & $\num{457.02 \pm 31.26}$ \\ \hline
\end{tabular}%
}
\caption{Ergebnisse des gewichteten linearen $\mathrm{\chi}^2$-Fits zur Bestimmung der Abbrems­spannung. Die hier gezeigten Werte stammen aus Tabelle \ref{tab:578_second}.}
\label{tab:578_second_chi2}
\end{table}






















%=======================================================================================================================================================================================================
\subsection*{Balmer-Serie}
\begin{table}[H]
\centering
\resizebox{\columnwidth}{!}{%
  \begin{tabular}{|c|c|c|c|c|c|c|c|}
    \hline
    \multicolumn{8}{|c|}{Spektrallinie Hg} \\ \hline
   % \multicolumn{8}{|c|}{1.\ Ordnung} \\ \hline
    $\omega_B$ [\si{\degree}] & $\Delta\omega_B$ [\si{\degree}] 
  & Spektrallinien & $\omega_G$ [\si{\degree}] & $\Delta\omega_G$ [\si{\degree}] 
  & Farbe            & Dicke $d$ [Skt] & $\Delta d$ [Skt] \\ \hline
    145,0 & 0,5 &        & 48,0  & 0,5 & violett       & 4  & 1 \\ \hline
    145,0 & 0,5 &        & 49,0  & 0,5 & violett       & 2  & 1 \\ \hline
    145,0 & 0,5 &        & 49,5  & 0,5 & violett       & 2  & 1 \\ \hline
    145,0 & 0,5 &        & 50,5  & 0,5 & violett/blau  & 3  & 1 \\ \hline
    145,0 & 0,5 &        & 51,0  & 0,5 & violett/blau  & 3  & 1 \\ \hline
    145,0 & 0,5 &        & 51,0  & 0,5 & blau          & 4  & 1 \\ \hline
    145,0 & 0,5 &        & 55,5  & 0,5 & türkis        & 1  & 0,1 \\ \hline
    145,0 & 0,5 &        & 61,0  & 0,5 & grün          & 3  & 1 \\ \hline
    145,0 & 0,5 &        & 64,0  & 0,5 & gelb          & 5  & 1 \\ \hline
    145,0 & 0,5 &        & 64,5  & 0,5 & gelb          & 5  & 1 \\ \hline
    145,0 & 0,5 &        & 69,0  & 0,5 & rot           & 5  & 1 \\ \hline
    135,0 & 0,5 &        & 68,0  & 0,5 & grün          & 8  & 1 \\ \hline
    135,0 & 0,5 &        & 71,0  & 0,5 & gelb          & 10 & 1 \\ \hline
    135,0 & 0,5 &        & 71,5  & 0,5 & gelb          & 11 & 1 \\ \hline
    155,0 & 0,5 &        & 61,0  & 0,5 & rot           & 1  & 0,1 \\ \hline
    155,0 & 0,5 &        & 61,5  & 0,5 & rot           & 1  & 0,1 \\ \hline
    155,0 & 0,5 &        & 62,5  & 0,5 & rot           & 2  & 1 \\ \hline
  \end{tabular}%
}
\caption{Spektrallinien der Hg-Dampflampe 1.\ Ordnung, gemessen an den Winkelpositionen und beobachteter Farbe. Hierbei ist $d$ die Dicke der Spektrallinien (in Strichpunkten), 
$\omega_B$ der Winkel der optischen Bank und $\omega_G$ der Winkel des Gitters. } 
\label{tab:Hg_lines}
\end{table}
%=======================================================================================================================================================================================================
\begin{table}[H]
\centering
\resizebox{\columnwidth}{!}{%
  \begin{tabular}{|c|c|c|c|c|c|c|c|}
    \hline
    \multicolumn{8}{|c|}{Spektrallinie H/Deuterium} \\ \hline
    %\multicolumn{8}{|c|}{1.\ Ordnung} \\ \hline
    $\omega_B$ [\si{\degree}] & $\Delta\omega_B$ [\si{\degree}]
  & Spektrallinien & $\omega_G$ [\si{\degree}] & $\Delta\omega_G$ [\si{\degree}]
  & Farbe            & $d$ [Skt] & $\Delta d$ [Skt] \\ \hline
    145,0 & 0,5 & & 51,0 & 0,5 & violett & 3 & 1 \\ \hline
    145,0 & 0,5 & & 55,5 & 0,5 & türkis  & 1 & 0,1 \\ \hline
    155,0 & 0,5 & & 61,5 & 0,5 & rot     & 1 & 0,1 \\ \hline
  \end{tabular}%
}
\caption{Spektrallinien der H/Deuterium-Lampe in erster Ordnung. Hierbei ist $d$ die Dicke der Spektrallinien (in Strichpunkten), $\omega_B$ der Winkel der Blende und $\omega_G$ der Beugungswinkel.}
\label{tab:H_D_lines}
\end{table}
