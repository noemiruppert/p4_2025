\chapter{Fazit}

Im ersten Teil des Experiments gelang die Nachweisung des photoelektrischen Effekts durch Bestrahlung der Photozelle mit einer Quecksilberdampflampe. Es konnten unterschiedliche Grenzspannungen ermittelt werden. Durch Auftragen der Randspannungen als Funktion der Frequenz wurde eine Einsteingerade ermittelt, aus deren Steigung sich die Planck-Konstante in der Form \( h = ( \pm ) \times 10^{-34} \) J s ableiten lässt. Der Achsenabschnitt der y-Achse ergab die Austrittsarbeit \( W_A = ( \pm ) \) eV. Somit konnte die Plancksche Konstante mit einer Genauigkeit von etwa  \% bestimmt werden.

Im zweiten Teil des Experiments wurde die Balmer-Reihe analysiert. Die Gitterkonstante des holographischen Gitters wurde zu \(g = ( \pm ) \) nm bestimmt, was innerhalb der Fehlergrenzen mit dem Literaturwert übereinstimmt. Die für die Balmer-Linien ermittelten Wellenlängen zeigen eine gute Übereinstimmung mit den Literaturwerten. Darüber hinaus wurde die Isotopenspaltung nachgewiesen. Auch die mit der CCD-Kamera gemessenen Wellenlängenverschiebungswerte stimmen gut mit bekannten Literaturwerten überein. Darüber hinaus wurde die Rydberg-Konstante mit \(R = ( \pm ) \ \) m\(^{-1}\) bestimmt, was eine sehr geringe/etwas/sehr Abweichung vom Literaturwert darstellt. Schließlich wurde die Planck-Konstante neu definiert und \( h = ( \pm ) \ mal 10^{-34} \) J s erhalten, was ein noch genaueres Ergebnis als der photoelektrische Effekt ist und somit näher am Literaturwert liegt.