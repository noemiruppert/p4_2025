\chapter{Einleitung}
Das Verständnis fundamentaler Quellen elektronischen Rauschens ist für die experimentelle Physik von zentraler Bedeutung. 
Zu den wichtigsten Rauschmechanismen zählen das Johnson-Rauschen (thermisches Rauschen) und das Schrotrauschen, die auf grundlegend verschiedenen mikroskopischen Prozessen beruhen. 
Johnson-Rauschen entsteht durch die thermisch bedingte, zufällige Bewegung von Ladungsträgern in Widerständen. 
Das Schrotrauschen hingegen resultiert aus der diskreten Natur der Elektronen, die zu statistischen Fluktuationen im Strom führt, wenn Ladungsträger unabhängig voneinander potenzielle Barrieren überwinden. 
Beide Rauscharten sind intrinsisch stochastisch, weisen im relevanten Frequenzbereich spektral konstante Leistungen (weißes Rauschen) auf und spiegeln zentrale Konzepte der statistischen Physik wider.

Johnson-Rauschen wird durch die Nyquist'sche Formel beschrieben, welche die mittlere quadratische Spannungsfluktuation über einem ohmschen Widerstand mit dessen Temperatur, dem Widerstandswert sowie der effektiven Messbandbreite verknüpft. 
Das Rauschen ist proportional zur Temperatur und zum Widerstand und erlaubt die experimentelle Bestimmung der Boltzmann-Konstanten~$k_B$. 
Das Schrotrauschen folgt der Schottky'schen Formel, die eine mittlere quadratische Stromfluktuation proportional zum Gleichstrom~$I_{\mathrm{dc}}$ und zur Bandbreite vorhersagt. 
Durch Messung des Schrotrauschens lässt sich die Elementarladung~$e$ experimentell bestimmen und die Gültigkeit eines poissonverteilten Transportprozesses überprüfen.

Ziel des vorliegenden Versuchs ist die quantitative Untersuchung von Johnson- und Schrotrauschen durch rauschoptimierte Verstärkerschaltungen, spektrale Filterung und RMS-Signalverarbeitung. 
Durch Variation des Widerstandswerts, des Gleichstroms und der effektiven Bandbreite sollen die theoretischen Zusammenhänge experimentell überprüft werden. 
Die präzise Kalibrierung der Signalpfade und die Berücksichtigung von Hintergrundrauschen ermöglichen dabei die Bestimmung fundamentaler Naturkonstanten und vertiefen das Verständnis für elektronische Fluktuationen in physikalischen Messsystemen.

