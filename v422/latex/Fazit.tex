\chapter{Fazit}
In diesem Experiment wurde der Umgang mit einem Rastertunnelmikroskop erfolgreich erlernt. Eine Goldprobe und eine HOPG-Probe wurden analysiert. Ziel der Untersuchung der Goldprobe war die Erstellung hochwertiger Bilder, die die Oberflächenstruktur der Probe detailliert beschreiben – ein Ziel, das erfolgreich erreicht/nicht erfolgreich erreicht wurde. Anschließend wurde die HOPG-Probe analysiert, bei welchen eine atomare Auflösung erreicht werden sollte. Diese Analyse verlief erfolgreich/nicht erfolgreich, sodass die durchschnittlichen Bindungswinkel und Atomabstände der Kohlenstoffatome aus den erhaltenen Bildern bestimmt werden konnten/nur annähernd bestimmt werden konnten.
Die Messungen ergaben einen durchschnittlichen Bindungswinkel von ( ± )° und einen durchschnittlichen Atomabstand von ( ± ) [Einheit]. Obwohl der ermittelte Bindungswinkel weitgehend mit den Literaturwerten übereinstimmt/nicht übereinstimmt, weist der durchschnittliche Atomabstand eine signifikante/kleine Abweichung auf. Eine mögliche Fehlerquelle ist eine ungenaue Kalibrierung der Piezoelemente im STM. Eine perspektivische Verzerrung, die durch die Neigung der Gitterebene der Probe verursacht wird, könnte ebenfalls zu Abweichungen geführt haben. Daher sind zusätzliche Kalibrierungsmessungen mit dem Gerät erforderlich, um die Genauigkeit der Ergebnisse zu verbessern.