\chapter{Fazit}
Im Verlauf dieses Praktikums wurde die Funktionsweise eines Rastertunnelmikroskops praxisnah vermittelt, indem zunächst eine Goldprobe untersucht und anschließend versucht wurde, die atomare Struktur von HOPG abzubilden. Bei der Goldprobe ließ sich ein stabiler Tunnelkontakt herstellen, und charakteristische ''wolkenartige'' Strukturen eines Fermi-Gases waren im ''Constant-Current-Mode'' deutlich erkennbar. Die Bildqualität entsprach dabei den Erwartungen aus der Praktikumsanleitung; trotz einzelner Kratzer im Mikrometerbereich konnten reproduzierbare Höhen- und Stromkarten erstellt werden. Damit wurde nachgewiesen, dass das STM grundlegend korrekt arbeitet und sich zur Erfassung nanometergroßer Topographien eignet.

Bei der HOPG-Probe war das Ziel, die hexagonale Kohlenstoffgitterstruktur auf atomarer Skala abzubilden, jedoch nicht vollständig erreichbar. Zwar konnten in großen Bereichen ($\SI{200}{\nm} \times \SI{200}{\nm}$) ebene Regionen identifiziert werden und Scans in Teilbereichen von bis zu $\SI{10}{\nm} \times \SI{10}{\nm}$ im ''Constant-Current-Mode'' problemlos Daten liefern, doch in den $\SI{4}{\nm} \times \SI{4}{\nm}$- und $\SI{5}{\nm} \times \SI{5}{\nm}$-Bereichen zeigte sich kein klares periodisches Strommuster. Stattdessen traten nur inkonsistente helle Streifen und sporadische Maxima auf, die auf Reglerdynamik, Vibrationen und unzuverlässige ''Approach''-/''Withdraw''-Mechanismen zurückzuführen sind. Die wiederholten Spitzenwechsel - obwohl stets ein Tunnelstrom nachgewiesen werden konnte - sowie das ständige Nachschärfen infolge fortwährender Kollisionen erschwerten eine stabile Abtastung zusätzlich. Letztlich ließ sich keine eindeutige hexagonale Gitterstruktur auflösen.

Aus der Analyse aller Einflussfaktoren ergibt sich, dass künftige HOPG-Messungen insbesondere von einer strengeren Spitzenvorbereitung, einer systematischen Parameteroptimierung und einer verbesserten mechanischen Stabilität profitieren würden. Der Einsatz vorgefertigter, scharfer Pt/Ir- oder Wolframspitzen, das Kalibrieren des Scanbereichs mit gut charakterisierten Referenzproben und ein verlässlicher Closed-Loop-''Approach'' würden Auflösungsverluste deutlich verringern. Ebenso essenziell sind die Minimierung akustischer und thermischer Störungen sowie längere Gleichgewichtsphasen nach jedem Heranfahrvorgang, um den Drift der Spitze zu reduzieren. Unter diesen verbesserten Bedingungen ist zu erwarten, dass sich das HOPG-Gitter in einem subnanometrischen Scanbereich zuverlässig auflösen lässt. Insgesamt verdeutlichte das Experiment, wie empfindlich STM-Messungen auf kleinste Veränderungen reagieren, und bietet nun eine fundierte Grundlage für optimierte Atomsondenmessungen in der Zukunft.


















%In diesem Experiment wurde der Umgang mit einem Rastertunnelmikroskop erfolgreich erlernt. Eine Goldprobe und eine HOPG-Probe wurden analysiert. Ziel der Untersuchung der Goldprobe war die Erstellung hochwertiger Bilder, die die Oberflächenstruktur der Probe detailliert beschreiben – ein Ziel, das erfolgreich erreicht/nicht erfolgreich erreicht wurde. Anschließend wurde die HOPG-Probe analysiert, bei welchen eine atomare Auflösung erreicht werden sollte. Diese Analyse verlief erfolgreich/nicht erfolgreich, sodass die durchschnittlichen Bindungswinkel und Atomabstände der Kohlenstoffatome aus den erhaltenen Bildern bestimmt werden konnten/nur annähernd bestimmt werden konnten.
%Die Messungen ergaben einen durchschnittlichen Bindungswinkel von ( ± )° und einen durchschnittlichen Atomabstand von ( ± ) [Einheit]. Obwohl der ermittelte Bindungswinkel weitgehend mit den Literaturwerten übereinstimmt/nicht übereinstimmt, weist der durchschnittliche Atomabstand eine signifikante/kleine Abweichung auf. Eine mögliche Fehlerquelle ist eine ungenaue Kalibrierung der Piezoelemente im STM. Eine perspektivische Verzerrung, die durch die Neigung der Gitterebene der Probe verursacht wird, könnte ebenfalls zu Abweichungen geführt haben. Daher sind zusätzliche Kalibrierungsmessungen mit dem Gerät erforderlich, um die Genauigkeit der Ergebnisse zu verbessern.

