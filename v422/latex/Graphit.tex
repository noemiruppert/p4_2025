Bevor die HOPG-Probe mit atomarer Auflösung analysiert werden kann, wird ihre grobe Oberflächenstruktur zunächst mit einem USB-Lichtmikroskop untersucht. Abbildung !!!ADDREF!!!! zeigt den Zustand der Probe (HOPG) nach dem Entfernen einer Schicht. Es ist zu erkennen, dass die oberste Graphitschicht keine gleichmäßig ebene Oberfläche aufweist, sondern aus mehreren plattenartigen Segmenten besteht, die in unterschiedlichen Winkeln zueinander ausgerichtet sind. Dies wird durch die unterschiedlichen Reflexionswinkel der einzelnen Bereiche deutlich.

Um eine möglichst glatte Oberfläche für die Analyse mit einem Rastertunnelmikroskop zu erzeugen, wird die oberste Graphitschicht erneut mit Klebeband entfernt. Nach dieser Behandlung erscheint die Probe optisch glatt, das mit dem USB-Mikroskop aufgenommene Bild !!!!ADDREF!!! zeigt jedoch noch eine gewisse Segmentierung (!!!!!CONFIRM OR NOT THIS OBSERVATION!!!!). Für RTM-Untersuchungen ist jedoch nur eine ebene Oberfläche über wenige Quadratnanometer erforderlich, sodass der hergestellte Zustand als ausreichend geeignet gilt.

Im nächsten Schritt wird die HOPG-Probe in den Probenhalter des Rastertunnelmikroskops eingesetzt. Zusätzlich wird eine neue Spitze für die Untersuchung vorbereitet und in die dafür vorgesehene Klemme des STM eingesetzt. Eine vergrößerte Abbildung der Spitze ist in Abbildung !!!!ADDREF!!! dargestellt.


Nach erfolgreicher Positionierung von Probe und Spitze erfolgt die Annäherung zunächst mit "advance" und anschließend mit "approach". Sobald die Steuerelektronik den Tunnelkontakt registriert, kann die Messung gestartet werden. Es wurden mehrere Bilder mit unterschiedlichen Vergrößerungen aufgenommen.

Das RTM wurde zunächst im "constant current mode" betrieben, um Kollisionen der Spitze mit möglichen Oberflächenunebenheiten zu vermeiden. Der "constant height mode" wurde später für hochauflösende Detailbilder auf atomarer Ebene verwendet.

Abbildung !!!ADDREF!!! zeigt ein Bild einer Graphitoberfläche mit einem Raster von !!!ADDSCALING!!! Kantenlänge. Da es im "constant current mode" aufgenommen wurde, enthält das Höhenbild !!!WHERE!!! topografische Informationen. Der dargestellte Ausschnitt zeigt drei weitgehend planare Segmente der Graphitoberfläche, die bereits im Lichtmikroskopbild makroskopisch sichtbar waren !!!ADDREF!!!. Ihre Kanten sind scharf, was im Strombild !!!WHERE!!! besonders deutlich wird, da der PID-Regler eine schnelle Änderung des Tunnelstroms erfasste. Insgesamt zeigt das Strombild, dass der Regelkreis einen weitgehend konstanten Tunnelstrom aufrechterhält, was auf eine präzise Einstellung der Regelparameter hindeutet. !!!CONFIRM/DENYOBSERVATION!!!

Das Höhenbild in !!!ADDREF!!! zeigt deutlich, dass der zu beobachtende Probenbereich für die atomare Auflösung geeignet ist, da das zentrale Segment eine nahezu flache Struktur im Nanometerbereich aufweist.!!!CONFIRM/DENYOBSERVATION!!! 

Zur Bestätigung !!!DIDWEDOTHAT???!!!!!! wurde zusätzlich ein kleineres Ausschnittsbild von !!!!ADDSCALING!!! aufgenommen. Das Ergebnis dieser Messung ist in !!!!ADDREF!!!! dargestellt.

Es gibt keine größeren Unregelmäßigkeiten im gescannten Gebiet. Lediglich vereinzelte Abweichungen vom Mittelwert sind sowohl im Höhenbild als auch im Strömungsbild erkennbar. Zur quantitativen Analyse ist die Verteilungsfunktion der Höhenwerte in !!!ADDREF-HÖHENPROFIL!!! dargestellt. Die Daten können durch eine Gauß-Funktion mit Mittelwert !!!!!!!!!!ADD-PARAMETERS-GAUSSFIT mu =  und Standardabweichung sigma= !!!!!!!!!!!! modelliert werden.

Da der Bereich \( \sigma \) sehr klein ist, kann angenommen werden, dass die Aufnahme im "constant height mode" durchgeführt werden kann, ohne die Spitze oder die Probe zu beschädigen. Um in diesen Modus zu wechseln, wird der PID-Regler mit den Parametern P = 0, I = 4, D = 0 eingestellt.

Die nächste Aufnahme wird mit einer Scanbreite von !!!!ADDSCALE!!!!!! durchgeführt. Abbildung !!!!ADDREF!!!!! zeigt die entsprechenden Messergebnisse des RTM. Da in dieser Aufnahme der "constant height mode" verwendet wurde, enthält das aktuelle Bild nun topografische Informationen. Das Höhenbild zeigt nur Stellen, an denen die feste Komponente des PID-Reglers kleine Höhenanpassungen vorgenommen hat, um große Bodenunebenheiten oder Gefälle.

!!!!ADD PICS AND COMMENTS EXPLAINING THAT YOU CAN'T SEE SHIT AND MAYBE SOME PICS OF WHAT IT SHOULD LOOK LIKE THEORETICALLY AND HOW WE WOULD DETERMINE THE ANGLE AND THE DISTANCE WITH GOOD PICS!!!!!!