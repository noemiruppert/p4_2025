\chapter{Untersuchung mit dem Lichtmikroskop und Bestimmung des Maßstabs}
Vor der Analyse mit dem Rastertunnelmikroskop wird der Zustand der Proben mit einem USB-Lichtmikroskop dokumentiert. Um den Maßstab der mikroskopischen Bilder zu bestimmen, erfolgt ein Vergleich mit einem TEM-Gitter, das durch regelmäßig angeordnete kleine rechteckige Öffnungen gekennzeichnet ist. Bevor dieser Vergleich durchgeführt werden kann, muss das TEM-Gitter zunächst mit einem Maßstab kalibriert werden.\\

%add picture of the grid under the microscope
Das TEM-Gitter wird zusammen mit einem transparenten Maßstab mit einem Mikroskop abgebildet (siehe Abbildung %addref). 
Der Abstand der Skalenstriche beträgt d = 0,5 mm. %nachprüfen
Anschließend wird die Anzahl der Pixel p zwischen den Skalenstrichen in einem Grafikprogramm gemessen. Der Umrechnungsfaktor k lässt sich über die Beziehung $b = k\cdotp$ ermitteln. Die Messwerte der in Abbildung %addref 
markierten Abstände sind in Tabelle %addref
aufgeführt.\\ %add table with calculated values if there are any

Daraus ergibt sich ein varianzgewichteter Mittelwert von $k = ( \pm ) \mu m/px$. Anschließend wird der Abstand l ermittelt, der genau ?? Gitterzellen umfasst. Daraus ergibt sich eine Messlänge von $(\pm)$ Pixeln, was mit dem zuvor ermittelten Umrechnungsfaktor zu $l = (\pm) \mu m$ führt. Der Abstand zwischen zwei Gitterzellen beträgt somit $a = (\pm) \mu m$.
Diese Informationen können zur Kalibrierung der Bilder des USB-Mikroskops verwendet werden. Nach jeder Probenaufnahme wird das TEM-Gitter auf die Probenplatte gelegt und auf die gleiche Fokusebene des Mikroskops justiert. Anschließend wird ein Foto des TEM-Gitters aufgenommen und der Abstand zwischen den Gitterzellen gemessen. Der Maßstab wird durch Vergleich mit dem zuvor ermittelten Gitterzellenabstand d1 berechnet. Im Anhang befinden sich Bilder des TEM-Gitters in verschiedenen Fokusebenen.