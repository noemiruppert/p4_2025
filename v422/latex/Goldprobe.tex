\chapter{Goldprobe}
Alle mit dem RTM aufgenommenen Bilder wurden mithilfe der frei verfügbaren Software „Gwyddion“ nachbearbeitet.
Die Software ermöglicht eine Vielzahl von Bildbearbeitungsfunktionen, darunter Rauschunterdrückung, Filterung und Kontrastanpassung. Diese Schritte sind entscheidend, um die Qualität der Bilder zu verbessern und die atomaren Strukturen klarer darzustellen.\\
Im ersten Versuchsteil wird eine Goldkugelprobe untersucht. Um die Leitfähigkeit von der Probe sicherzustellen, wurde Gold auf Silizium aufgedampft. Die Oberfläche wurde unter dem Mikroskop vergrößert (siehe Abbildung %addref
). Das rechte Bild zeigt einen vergrößerten Ausschnitt des linken Bildes. Zu beachten ist, dass die verwendete Probe nicht völlig eben und kratzfrei ist. Für das Experiment stellt dies jedoch kein Problem dar, da genügend möglichst ebene und kratzfreie Bereiche vorhanden sind. Für diesen Teil des Experiments wurde die Spitze aus Abbildung %addref
 verwendet.
 %add picture of the gold probe
\begin{figure}[H]
    \centering
    \includegraphics[width=0.8\textwidth]{}
    \caption{Goldkugelprobe unter dem Mikroskop. Links: Gesamtansicht, rechts: vergrößerter Ausschnitt.}
    \label{fig:goldprobe}
\end{figure}
Die Messung wurde im constant current mode durchgeführt, wobei eine Tunnelspannung von $U = 1\,\text{V}$ und ein Strom von $I = 1\,\text{nA}$ eingestellt wurden. Die PID-Regelparameter waren $P = 1000$, $I = 2000$ und $D = 0$. Der Scanbereich betrug $???,\text{nm} \times ???\text{nm}$. Das resultierende Bild zeigt die topografischen Eigenschaften der Goldprobe mit hoher lateraler und vertikaler Auflösung. Die atomaren Strukturen sind deutlich erkennbar, was auf die hohe Präzision des RTM hinweist.\\