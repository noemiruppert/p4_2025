\chapter{Goldprobe}
Alle mit dem RTM aufgenommenen Bilder wurden mithilfe der frei verfügbaren Software „Gwyddion“ nachbearbeitet.
Die Software ermöglicht eine Vielzahl von Bildbearbeitungsfunktionen, darunter Rauschunterdrückung, Filterung und Kontrastanpassung. Diese Schritte sind entscheidend, um die Qualität der Bilder zu verbessern und die atomaren Strukturen klarer darzustellen.\\
Im ersten Versuchsteil wird eine Goldkugelprobe untersucht. Um die Leitfähigkeit von der Probe sicherzustellen, wurde Gold auf Silizium aufgedampft. Die Oberfläche wurde unter dem Mikroskop vergrößert (siehe Abbildung !!!!!!!ADD REFERENCE!!!!!!). 
Das rechte Bild zeigt einen vergrößerten Ausschnitt des linken Bildes. Zu beachten ist, dass die verwendete Probe nicht völlig eben und kratzfrei ist. Für das Experiment stellt dies jedoch kein Problem dar, da genügend möglichst ebene und kratzfreie Bereiche vorhanden sind. Für diesen Teil des Experiments wurde die Spitze aus Abbildung !!!!!!!ADD REFERENCE!!!!!! verwendet.

 ADD PICTURE!!!!!!!!!!!!!!!
%\begin{figure}[H]
 %   \centering
  %  \includegraphics[width=0.8\textwidth]{}
   % \caption{Goldkugelprobe unter dem Mikroskop. Links: Gesamtansicht, rechts: vergrößerter Ausschnitt.}
   % \label{fig:goldprobe}
%\end{figure}
Die Messung wurde im constant current mode durchgeführt, wobei eine Tunnelspannung von $U = 1\,\text{V}$ und ein Strom von $I = 1\,\text{nA}$ eingestellt wurden. Die PID-Regelparameter waren $P = 1000$, $I = 2000$ und $D = 0$. Der Scanbereich betrug $???,\text{nm} \times ???\text{nm}$. Das resultierende Bild zeigt die topografischen Eigenschaften der Goldprobe mit hoher lateraler und vertikaler Auflösung. Die atomaren Strukturen sind deutlich erkennbar, was auf die hohe Präzision des RTM hinweist.\\

Die verarbeiteten Bilder sind in den Abbildungen !!!!!!!ADD REFERENCE!!!!!! dargestellt. Sie zeigen das Höhenprofil sowohl im 2D- als auch im 3D-Format, ergänzt durch die entsprechenden Strombilder.
%add the pictures taken with the microscope on different scales 

Die theoretische Erwartung wurde durch die Untersuchung der Goldprobe bestätigt. Die Goldprobe weist keine regelmäßige Oberflächenstruktur auf. Dies entspricht den Erwartungen, da sich die Elektronen in Gold wie ein frei bewegliches Elektronengas (Fermigas) mit wolkenartiger Verteilung verhalten. Diese Eigenschaft ist typisch für Metalloberflächen, bei denen Elektronentransfer zu einer nahezu kontinuierlichen und homogenen Tunnelstromverteilung führt – im Gegensatz zur lokalen, geordneten Strukturen. Auffällig ist, dass die Elektronenverteilung in hellen Bereichen deutlich größer ist als in dunklen. Darüber hinaus ist klar, dass der PID-Regelkreis mit geeigneten Parametern arbeitet, da die Strombilder insgesamt einen geringen Kontrast aufweisen, was auf eine geringe Stromschwankung hindeutet. %verify this observation

Die Bilder weisen möglichst wenige Fehler und Verzerrungen auf, die durch eine optimale PID-Regelung oder eine zu hohe Scangeschwindigkeit verursacht werden können. Im Experiment zeigte sich, dass langsameres Scannen zu einer höheren Auflösung der erfassten Struktur führte. Der Nachteil dieser Methode ist die längere Aufnahmezeit, die mögliche Bildfehler, beispielsweise durch Geräusche oder Bewegungen, verstärken kann. Daher ist die Wahl einer angemessenen Scangeschwindigkeit für die Messung entscheidend. Trotz der geringen Bildfehler kann festgestellt werden, dass die Bilder ihren Zweck erfüllen und eine repräsentative Darstellung der Goldprobe liefern.