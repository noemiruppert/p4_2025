\chapter{Einleitung}

In diesem Experiment wird ein Rastertunnelmikroskop (RTM, engl. STM - Scanning Tunneling Microscope) zur Untersuchung der Oberflächenstruktur eingesetzt. 
Mit Pt-Ir-Spitzen werden Bilder einer Goldprobe und einer Probe aus hochorientiertem pyrolytischem Graphit (HOPG) aufgenommen. Ziel ist es, für die HOPG-Probe eine atomare Auflösung zu erreichen, um die Gitterkonstanten und Winkel der hexagonalen Struktur der Kohlenstoffatome zu bestimmen. Ziel des Experiments ist es, das Verständnis der Funktionsweise des Rastertunnelmikroskop zu vertiefen, verschiedene Oberflächenstrukturen zu analysieren und den zuverlässigen Umgang mit der Mikroskop-Steuerungssoftware zu erlernen.