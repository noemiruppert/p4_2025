\chapter{Einleitung}

In diesem Experiment wird ein Rastertunnelmikroskop (RTM) zur Untersuchung der Oberflächenstruktur eingesetzt. Mit speziell angefertigten Pt-Ir-Spitzen werden Bilder einer Goldprobe und einer Probe aus hochorientiertem pyrolytischem Graphit (HOPG) aufgenommen. Ziel ist es, für die HOPG-Probe eine atomare Auflösung zu erreichen, um die Bindungswinkel und Gitterabstände der Kohlenstoffatome aus den erhaltenen Bildern zu bestimmen. Ziel des Experiments ist es, das Verständnis der Funktionsweise des RTM zu vertiefen, verschiedene Oberflächenstrukturen zu analysieren und den zuverlässigen Umgang mit der Mikroskop-Steuerungssoftware zu erlernen.